\subsubsection{Soluciones}

\textbf{1.} Tableau Final
\begin{center}

\begin{tabular}{ c c | c | c | c | c | c | c | c }
		&	& $x_1$ & $x_2$ & $x_3$ & $s_1$ & $s_2$ & $s_3$ &\\
Base & $c_j$	& 5000	& 9500	& 15000	&   0	& 0 		&   0	& $b_j$\\
\hline
$x_3$ & 15000 & 0 & 1/3 & 1 & 1/3 & 0 & -1/3 & 10/3\\
$S_2$ & 0	  & 0  & -4/3 & 0 & 1/3 & 1 & -4/3 & 70/3\\
$x_1$ & 5000  & 1  & 11/3 & 0 & -2/3 & 0 & 5/3 & 10/3\\
\hline
& $z_j$		& 3000  & 23333.33  & 15000 & 1666.66 & 0 & 3333.33  & 66666.66\\
& $c_j - z_j$	& 0  & -13833.33  & 0 & -1666.66 & 0 & -3333.33 &\\
\end{tabular}

\end{center}

Debido a que sólo es 0 el precio sombra de $x_1$, $x_3$ y $S_2$, no existen soluciones alternativas.\\\\
$b_j$: cantidad de kilos a producir.\\
Solución óptima: máxima cantidad de dinero a ganar.\\
$c_j - z_j$: reducción de ganancias.
\\

\textbf{2.} $x_2$ puede variar entre $-\infty$ y $13333.33$ y no varía la solución óptima.
\begin{itemize}
\item El precio puede ser $\$12000$, ya que: $12000 - 1666*2 - 2*0 - 2*3333 = 2002$
\item El precio mínimo de las almendras debiese ser: $x=1666*2+2*0+2*3333$, osea $x=9998$
\end{itemize}


\textbf{3.} $3954.54 \leq x_1 \leq 7500$\\
El menor precio es $\$3954.54 \approx 13955$, considerando mantener el nivel de producción.

\textbf{4.} La solución obtenida luego de agregar esta restricción se mantiene. Esto debido a que la restricción no modifica la región factible o al menos no excluye el punto extremo óptimo actual.
