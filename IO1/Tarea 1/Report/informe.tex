\documentclass[12pt,letterpaper]{article}
\usepackage[utf8]{inputenx} %Codificacion del texto (ISO Latin1 encoding)

\usepackage{fancyhdr} %Permite acomodar a tu gusto la parte de arriba y
% abajo del documento
\usepackage[spanish]{babel} %Permite definir el idioma del dcumento
\usepackage{graphicx} %Permite exportar imagenes en formato eps
\usepackage{url} %Tipo de fuente para correos y paginas
\usepackage{pgf}
\usepackage{fleqn}
\usepackage{amssymb}
\usepackage{amsmath}
\usepackage{fancyvrb}
\usepackage{makeidx}
\usepackage{colortbl} %Permite colocar colores a las tablas
\usepackage{booktabs}
\usepackage[final]{pdfpages}
%%%%%%%%%%
%Margenes%
%%%%%%%%%%
\parskip 1mm %Espacio entre parrafos

\setlength{\topmargin}{0pt}
\topmargin      0.5cm
\oddsidemargin	0.1cm  % Ancho Letter 21,59cm
\evensidemargin 0.5cm  % Alto  Letter 27,81cm
\textwidth	17cm%15.5cm
\textheight	21.0cm
\headsep	4 mm
\parindent	0.5cm
%%%%%%%%%%%%%%%%%%%%%%
%Estilo del documento%
%%%%%%%%%%%%%%%%%%%%%%
\pagestyle{fancyplain}

%%%%%%%%%%%%%%%%%%%%%%%%%%%%%%%%%%%%%%%%%%%
%Fancyheadings. Top y Bottom del documento%
%%%%%%%%%%%%%%%%%%%%%%%%%%%%%%%%%%%%%%%%%%%
% Recuerde que en este documento la portada del documento no posee
% numeracion, pero de igual manera llamaremos a esa primera pagina la numero
% 1, y la que viene la dos. Esto es para tener una idea de las que
% llamaremos pares e impares
\lhead{Investigación de Operaciones I} %Parte superior izquierda
\rhead{\bf \it Tarea 1} %Parte superior derecha
\lfoot{\it } %Parte inferior izquierda. \thepage indica
% el numero de pagina
\cfoot{} %Parte inferior central
\rfoot{\bf \thepage} %Parte inferior derecha
\renewcommand{\footrulewidth}{0.4pt} %Linea de separacion inferior

\newcommand{\primaria}[1]{
	\textbf{\underline{#1}}
}

\newcommand{\foranea}[1]{
	\textbf{\textsl{#1}}
}

\newcommand{\primyfor}[1]{
	\underline{\foranea{#1}}
}

\makeatletter
\newcommand\subsubsubsection{\@startsection {paragraph}{1}{\z@}%
                                   {-3.5ex \@plus -1ex \@minus -.2ex}%
                                   {1.5ex \@plus.2ex}%
                                   {\normalfont\bfseries}}
                       
                                
                                 
\newcommand\subsubsubsubsection{\@startsection {subparagraph}{1}{\z@}%
                                   {-3.5ex \@plus -1ex \@minus -.2ex}%
                                   {1.5ex \@plus.2ex}%
                                  
                                   {\normalfont\bfseries}}


\makeatother
 

\begin{document}
\title{Investigación de Operaciones I \\ \begin{Large}Tarea 1\end{Large}} 
\author{Victor Gonzalez (2.773.029-9)
\and Cesar Muñoz (2.973.053-0)}
\date{\today}
\maketitle


\section{Fábrica de Encurtidos Pep \& Nillo}
\subsection{Enunciado}
\input{./src/tarea1.1}
\subsection{Solución}
\subsubsection{Modelo de programación lineal}
\begin{description}
\item[Variables:]
\item $A_i$: Cantidad de kilogramos de encurtidos ácidos en el mes i, donde $i=[1,4]$
\item $D_i$: Cantidad de kilogramos de encurtidos dulces en el mes i, donde $i=[1,4]$
\item $E_i$: Cantidad de kilogramos de encurtidos de eneldo en el mes i, donde $i=[1,4]$
\end{description}

\begin{description}
\item[Función Objetivo:]
\item[] $min z =$
\item[] $1000*0.3A_i + \frac{35000}{40}*0.7*A_i+300A_i$ // costo de encurtidos ácidos
\item[] $400D_i + 1000D_i$ // costo de encurtidos dulces
\item[] $700E_i + 1000E_i$ // costo de encurtidos de eneldo 
\end{description}

\begin{description}
\item[Sujeto A:]
\item[] $A_1 \geq 105$ // "15 kg de ácidos el primer mes"
\item[] $D_2 + D_3 \geq 400$ // "400 kg de dulce entre el segundo y tercer mes"
\item[] $E_3 + E_4 \geq 100$ // "100 kg de eneldo entre los últimos 2 meses"
\item[] $\frac{A_i+D_i+E_i}{4} \geq 400$ // "se necesita una media mensual de 400[$\frac{kg}{mes}$]"
\item[] $A_i, D_i, E_i \geq 0.2(A_i+D_i+E_i)$ // producción de un encurtido no menor al 20\% del total del mes
\item[] $A_i, D_i, E_i \geq 0$ // naturaleza de las variables
\end{description}

\subsubsection{Solución con LP\_Solve}

\textbf{Código en LP\_Solve}
\begin{verbatim}
min:	661.25 A1 + 661.25 A2 + 661.25 A3 + 661.25 A4 +
	1400 D1 + 1400 D2 + 1400 D3 + 1400 D4 +
	1700 E1 + 1700 E2 + 1700 E3 + 1700 E4;

A1 >= 105;
D2 + D3 >= 400;
E3 + E4 >= 100;
A1 + A2 + A3 + A4 + D1 + D2 + D3 + D4 + E1 + E2 + E3 + E4 >= 1600;

A1 >= 0.2 A1 + 0.2 D1 + 0.2 E1;
A2 >= 0.2 A2 + 0.2 D2 + 0.2 E2;
A3 >= 0.2 A3 + 0.2 D3 + 0.2 E3;
A4 >= 0.2 A4 + 0.2 D4 + 0.2 E4;

D1 >= 0.2 A1 + 0.2 D1 + 0.2 E1;
D2 >= 0.2 A2 + 0.2 D2 + 0.2 E2;
D3 >= 0.2 A3 + 0.2 D3 + 0.2 E3;
D4 >= 0.2 A4 + 0.2 D4 + 0.2 E4;

E1 >= 0.2 A1 + 0.2 D1 + 0.2 E1;
E2 >= 0.2 A2 + 0.2 D2 + 0.2 E2;
E3 >= 0.2 A3 + 0.2 D3 + 0.2 E3;
E4 >= 0.2 A4 + 0.2 D4 + 0.2 E4;
\end{verbatim}

\newpage
\textbf{Resultado de LP\_Solve}
\begin{verbatim}
Value of objective function: 1.71176e+06

Actual values of the variables:
A1                            105
A2                            440
A3                            300
A4                              0
D1                             35
D2                            300
D3                            100
D4                              0
E1                             35
E2                            185
E3                            100
E4                              0
\end{verbatim}

\newpage

\section{Tostaduría Algún Maní}
Una fábrica produce tres tipos de fruto seco: maní almendra y pistacho. Cada uno se vende en paquetes de 1 kilo. Cada fruto pasa a trav\'es de tres procesos: limpieza, pelado y tostado. La fábrica dispone de 20 personas para tostar, 30 para limpiar y 10 para pelar. Los requerimientos para producir un kilo de fruto y su utilidad es la siguiente:

\begin{center}

\begin{tabular}{ c | c | c | c }
Requerimientos	& Man\'i & Almendra & Pistacho\\
\hline
Limpiar [pers/kilo]	& 1	  & 2   & 1\\
Pelado [pers/kilo]	& 1	  & 3   & 2\\
Tostado [pers/kilo]	& 1	  & 2   & 5\\
Utilidad [\$/kilo]	& 5000	  & 9500   & 15000\\
\end{tabular}

\end{center}

Definiendo las variables de decisi\'on $x_1$ ,$x_2$ y $x_3$ , que representan la cantida de kilos de maní, almendra y pistacho a producir respectivamente, se ha formulado el siguiente modelo de programación lineal y el tableau final:

máx $z = 5000x_1 + 9500x_2 + 15000x_3$\\
St.\\
$x_1 + 2x_2 + 5x_3 \leq 20$ (Tostar)\\
$x_1 + 2x_2 + x_3 \leq 30$ (Limpiar)\\
$x_1 + 3x_2 + 2x_3 \leq 10$ (Pelar)\\
$x_i \geq 0$

\begin{center}

\begin{tabular}{ c | c | c | c | c | c | c | c }
			& $x_1$ & $x_2$ & $x_3$ & $s_1$ & $s_2$ & $s_3$ &\\
Base $c_j$	& 5000	& 9500	& 15000	&   0	& 0 		&   0	& $b_j$\\
\hline
			& 0  & 1/3 & 1 & 1/3 & 0 & -1/3 & 10/3\\
			& 0  & -4/3 & 0 & 1/3 & 1 & -4/3 & 70/3\\
			& 1  & 11/3 & 0 & -2/3 & 0 & 5/3 & 10/3\\
\hline
$z_j$		&   &   &  & & & &\\
$c_j - z_j$	&   &   &  & & & &\\
\end{tabular}

\end{center}

A partir de lo anterior, responda cada una de las preguntas de forma independiente, es decir, si en alguna modificó el tableau, no considere dicho cambio en la próxima pregunta.\\

1. Complete el tableau final para obtener la solución óptima e interprete el valor de cada una de las variables que ahí aparecen. Indique si existe o no solución alternativa. Justifique.\\

2. Se cree que si el precio del kilo de almendra llega a 12000, es conveniente producirla. Está ud. de acuerdo? Justifique. ¿Cuál debiera ser el precio mínimo de las almendras?\\

3. Debido a una acumulación de maní en las bodegas, se ha iniciado una campaña de venta basándose en un precio muy bajo para el público. ¿Cuál debiera ser este precio si la idea es mantener el nivel de producción?\\

4. Si producto de una nueva tecnología incorporada al proceso, se necesita que cada kilo de fruto seco esté almacenado en una bodega con temperatura y humedad regulada según la siguiente tabla:

\begin{center}

\begin{tabular}{ c | c | c | c | c}
Requerimiento	& Man\'i & Almendra & Pistacho & Disponibilidad [hr]\\
\hline
Almacenamiento [Hr/kilo]	& 2	  & 5   & 8  & 35\\
\end{tabular}

\end{center}

¿Qué ocurre con la solución óptima encontrada? ¿Se mantiene? ¿Cambia? Argumente.





\input{./src/solucion1.2}
\newpage



\bibliographystyle{alpha}
\bibliography{bibbase}

% referencias

\end{document}
