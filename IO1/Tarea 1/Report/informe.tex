\documentclass[12pt,letterpaper]{article}
\usepackage[utf8]{inputenx} %Codificacion del texto (ISO Latin1 encoding)

\usepackage{fancyhdr} %Permite acomodar a tu gusto la parte de arriba y
% abajo del documento
\usepackage[spanish]{babel} %Permite definir el idioma del dcumento
\usepackage{graphicx} %Permite exportar imagenes en formato eps
\usepackage{url} %Tipo de fuente para correos y paginas
\usepackage{pgf}
\usepackage{fleqn}
\usepackage{amssymb}
\usepackage{amsmath}
\usepackage{fancyvrb}
\usepackage{makeidx}
\usepackage{colortbl} %Permite colocar colores a las tablas
\usepackage{booktabs}
\usepackage[final]{pdfpages}
%%%%%%%%%%
%Margenes%
%%%%%%%%%%
\parskip 1mm %Espacio entre parrafos

\setlength{\topmargin}{0pt}
\topmargin      0.5cm
\oddsidemargin	0.1cm  % Ancho Letter 21,59cm
\evensidemargin 0.5cm  % Alto  Letter 27,81cm
\textwidth	17cm%15.5cm
\textheight	21.0cm
\headsep	4 mm
\parindent	0.5cm
%%%%%%%%%%%%%%%%%%%%%%
%Estilo del documento%
%%%%%%%%%%%%%%%%%%%%%%
\pagestyle{fancyplain}

%%%%%%%%%%%%%%%%%%%%%%%%%%%%%%%%%%%%%%%%%%%
%Fancyheadings. Top y Bottom del documento%
%%%%%%%%%%%%%%%%%%%%%%%%%%%%%%%%%%%%%%%%%%%
% Recuerde que en este documento la portada del documento no posee
% numeracion, pero de igual manera llamaremos a esa primera pagina la numero
% 1, y la que viene la dos. Esto es para tener una idea de las que
% llamaremos pares e impares
\lhead{Investigación de Operaciones I} %Parte superior izquierda
\rhead{\bf \it Tarea 1} %Parte superior derecha
\lfoot{\it } %Parte inferior izquierda. \thepage indica
% el numero de pagina
\cfoot{} %Parte inferior central
\rfoot{\bf \thepage} %Parte inferior derecha
\renewcommand{\footrulewidth}{0.4pt} %Linea de separacion inferior

\newcommand{\primaria}[1]{
	\textbf{\underline{#1}}
}

\newcommand{\foranea}[1]{
	\textbf{\textsl{#1}}
}

\newcommand{\primyfor}[1]{
	\underline{\foranea{#1}}
}

\makeatletter
\newcommand\subsubsubsection{\@startsection {paragraph}{1}{\z@}%
                                   {-3.5ex \@plus -1ex \@minus -.2ex}%
                                   {1.5ex \@plus.2ex}%
                                   {\normalfont\bfseries}}
                       
                                
                                 
\newcommand\subsubsubsubsection{\@startsection {subparagraph}{1}{\z@}%
                                   {-3.5ex \@plus -1ex \@minus -.2ex}%
                                   {1.5ex \@plus.2ex}%
                                  
                                   {\normalfont\bfseries}}


\makeatother
 

\begin{document}
\title{Investigación de Operaciones I \\ \begin{Large}Tarea 1\end{Large}} 
\author{Victor Gonzalez (2.773.029-9)
\and Cesar Muñoz (2.973.053-0)}
\date{\today}
\maketitle


\section{Fábrica de Encurtidos Pep \& Nillo}
\subsection{Enunciado}
La fábrica de encurtidos Pep \& Nillo ofrece tres productos: encurtidos ácidos, dulces y encurtidos al eneldo. La empresa estima que necesita para su horizonte de planificación, que cubre los próximos 4 meses, una media de 400[kg/mes] de encurtidos, con, al menos, 105 kilos de encurtidos ácidos durante el primer mes, 400 kilos de encurtido dulce entre el segundo y tercer mes y 100 kilos de encurtidos al eneldo para cada uno de los ultimos dos meses.

Además, en cualquier mes la producción de cualquiera de los tres tipos de encurtido no puede bajar del 20 \% de la producción total de encurtidos del mes. La empresa sabe lo siguiente:\\

$\bullet$ Los encurtidos dulces y los encurtidos al eneldo se producen unicamente a base de pepinillos.\\

$\bullet$ Los encurtidos ácidos además incluyen zanahorias, cebollitas y otras especies hortícolas que se encargan ya pasteurizadas a otra compañía que sólo entrega barriles de 40 kilos, cada barril con un costo de \$$35000$.\\

$\bullet$ Un kilo de encurtido ácido se produce mezclando 300 gr de pepinillos y 700 gr de otros frutos.\\

$\bullet$ El costo de recolección, lavado y pasteurización de un kilo de pepinillo es de \$$1000$.\\

Además, los costos de producción de cada tipo de encurtido son los siguientes:

\begin{center}

\begin{tabular}{ c | c | c | c }
				& \'acido & dulce & al eneldo\\
\hline
Costo [\$/kilo]	& 300	  & 400   & 700\\
\end{tabular}

\end{center}

1. Formule un modelo de programación lineal que le permita satisfacer la demanda a mínimo costo.\\

2. Encuentre una solución con LP Solve y adjunte el código en la entrega.


\subsection{Solución}
\begin{description}
\item[Variables:]
\item $A_i$: Cantidad de kilogramos de encurtidos ácidos en el mes i, donde $i=[1,4]$
\item $D_i$: Cantidad de kilogramos de encurtidos dulces en el mes i, donde $i=[1,4]$
\item $E_i$: Cantidad de kilogramos de encurtidos de eneldo en el mes i, donde $i=[1,4]$
\end{description}

\begin{description}
\item[Función Objetivo]
\item $min z =$\\
\item 
\end{description}


\newpage

\section{Tostaduría Algún Maní}
Una fábrica produce tres tipos de fruto seco: maní almendra y pistacho. Cada uno se vende en paquetes de 1 kilo. Cada fruto pasa a trav\'es de tres procesos: limpieza, pelado y tostado. La fábrica dispone de 20 personas para tostar, 30 para limpiar y 10 para pelar. Los requerimientos para producir un kilo de fruto y su utilidad es la siguiente:

\begin{center}

\begin{tabular}{ c | c | c | c }
Requerimientos	& Man\'i & Almendra & Pistacho\\
\hline
Limpiar [pers/kilo]	& 1	  & 2   & 1\\
Pelado [pers/kilo]	& 1	  & 3   & 2\\
Tostado [pers/kilo]	& 1	  & 2   & 5\\
Utilidad [\$/kilo]	& 5000	  & 9500   & 15000\\
\end{tabular}

\end{center}

Definiendo las variables de decisi\'on $x_1$ ,$x_2$ y $x_3$ , que representan la cantida de kilos de maní, almendra y pistacho a producir respectivamente, se ha formulado el siguiente modelo de programación lineal y el tableau final:

máx $z = 5000x_1 + 9500x_2 + 15000x_3$\\
St.\\
$x_1 + 2x_2 + 5x_3 \leq 20$ (Tostar)\\
$x_1 + 2x_2 + x_3 \leq 30$ (Limpiar)\\
$x_1 + 3x_2 + 2x_3 \leq 10$ (Pelar)\\
$x_i \geq 0$

\begin{center}

\begin{tabular}{ c | c | c | c | c | c | c | c }
			& $x_1$ & $x_2$ & $x_3$ & $s_1$ & $s_2$ & $s_3$ &\\
Base $c_j$	& 5000	& 9500	& 15000	&   0	& 0 		&   0	& $b_j$\\
\hline
			& 0  & 1/3 & 1 & 1/3 & 0 & -1/3 & 10/3\\
			& 0  & -4/3 & 0 & 1/3 & 1 & -4/3 & 70/3\\
			& 1  & 11/3 & 0 & -2/3 & 0 & 5/3 & 10/3\\
\hline
$z_j$		&   &   &  & & & &\\
$c_j - z_j$	&   &   &  & & & &\\
\end{tabular}

\end{center}

A partir de lo anterior, responda cada una de las preguntas de forma independiente, es decir, si en alguna modificó el tableau, no considere dicho cambio en la próxima pregunta.\\

1. Complete el tableau final para obtener la solución óptima e interprete el valor de cada una de las variables que ahí aparecen. Indique si existe o no solución alternativa. Justifique.\\

2. Se cree que si el precio del kilo de almendra llega a 12000, es conveniente producirla. Está ud. de acuerdo? Justifique. ¿Cuál debiera ser el precio mínimo de las almendras?\\

3. Debido a una acumulación de maní en las bodegas, se ha iniciado una campaña de venta basándose en un precio muy bajo para el público. ¿Cuál debiera ser este precio si la idea es mantener el nivel de producción?\\

4. Si producto de una nueva tecnología incorporada al proceso, se necesita que cada kilo de fruto seco esté almacenado en una bodega con temperatura y humedad regulada según la siguiente tabla:

\begin{center}

\begin{tabular}{ c | c | c | c | c}
Requerimiento	& Man\'i & Almendra & Pistacho & Disponibilidad [hr]\\
\hline
Almacenamiento [Hr/kilo]	& 2	  & 5   & 8  & 35\\
\end{tabular}

\end{center}

¿Qué ocurre con la solución óptima encontrada? ¿Se mantiene? ¿Cambia? Argumente.





\subsubsection{Soluciones}

\textbf{1.} Tableau Final
\begin{center}

\begin{tabular}{ c c | c | c | c | c | c | c | c }
		&	& $x_1$ & $x_2$ & $x_3$ & $s_1$ & $s_2$ & $s_3$ &\\
Base & $c_j$	& 5000	& 9500	& 15000	&   0	& 0 		&   0	& $b_j$\\
\hline
$x_3$ & 15000 & 0 & 1/3 & 1 & 1/3 & 0 & -1/3 & 10/3\\
$S_2$ & 0	  & 0  & -4/3 & 0 & 1/3 & 1 & -4/3 & 70/3\\
$x_1$ & 5000  & 1  & 11/3 & 0 & -2/3 & 0 & 5/3 & 10/3\\
\hline
& $z_j$		& 3000  & 23333.33  & 15000 & 1666.66 & 0 & 3333.33  & 66666.66\\
& $c_j - z_j$	& 0  & -13833.33  & 0 & -1666.66 & 0 & -3333.33 &\\
\end{tabular}

\end{center}

Debido a que sólo es 0 el precio sombra de $x_1$, $x_3$ y $S_2$, no existen soluciones alternativas.\\\\
$b_j$: cantidad de kilos a producir.\\
Solución óptima: máxima cantidad de dinero a ganar.\\
$c_j - z_j$: reducción de ganancias.
\\

\textbf{2.} $x_2$ puede variar entre $-\infty$ y $13333.33$ y no varía la solución óptima.
\begin{itemize}
\item El precio puede ser $\$12000$, ya que: $12000 - 1666*2 - 2*0 - 2*3333 = 2002$
\item El precio mínimo de las almendras debiese ser: $x=1666*2+2*0+2*3333$, osea $x=9998$
\end{itemize}


\textbf{3.} $3954.54 \leq x_1 \leq 7500$\\
El menor precio es $\$3954.54 \approx 13955$, considerando mantener el nivel de producción.

\textbf{4.} La solución obtenida luego de agregar esta restricción se mantiene. Esto debido a que la restricción no modifica la región factible o al menos no excluye el punto extremo óptimo actual.

\newpage



\bibliographystyle{alpha}
\bibliography{bibbase}

% referencias

\end{document}
