\documentclass[12pt,letterpaper]{article}
\usepackage[utf8]{inputenx} %Codificacion del texto (ISO Latin1 encoding)

\usepackage{fancyhdr} %Permite acomodar a tu gusto la parte de arriba y
% abajo del documento
\usepackage[spanish]{babel} %Permite definir el idioma del dcumento
\usepackage{graphicx} %Permite exportar imagenes en formato eps
\usepackage{url} %Tipo de fuente para correos y paginas
\usepackage{pgf}
\usepackage{fleqn}
\usepackage{amssymb}
\usepackage{amsmath}
\usepackage{fancyvrb}
\usepackage{makeidx}
\usepackage{colortbl} %Permite colocar colores a las tablas
\usepackage{booktabs}
\usepackage[final]{pdfpages}
%%%%%%%%%%
%Margenes%
%%%%%%%%%%
\parskip 1mm %Espacio entre parrafos

\setlength{\topmargin}{0pt}
\topmargin      0.5cm
\oddsidemargin	0.1cm  % Ancho Letter 21,59cm
\evensidemargin 0.5cm  % Alto  Letter 27,81cm
\textwidth	17cm%15.5cm
\textheight	21.0cm
\headsep	4 mm
\parindent	0.5cm
%%%%%%%%%%%%%%%%%%%%%%
%Estilo del documento%
%%%%%%%%%%%%%%%%%%%%%%
\pagestyle{fancyplain}

%%%%%%%%%%%%%%%%%%%%%%%%%%%%%%%%%%%%%%%%%%%
%Fancyheadings. Top y Bottom del documento%
%%%%%%%%%%%%%%%%%%%%%%%%%%%%%%%%%%%%%%%%%%%
% Recuerde que en este documento la portada del documento no posee
% numeracion, pero de igual manera llamaremos a esa primera pagina la numero
% 1, y la que viene la dos. Esto es para tener una idea de las que
% llamaremos pares e impares
\lhead{Investigación de Operaciones I} %Parte superior izquierda
\rhead{\bf \it Tarea 3} %Parte superior derecha
\lfoot{\it } %Parte inferior izquierda. \thepage indica
% el numero de pagina
\cfoot{} %Parte inferior central
\rfoot{\bf \thepage} %Parte inferior derecha
\renewcommand{\footrulewidth}{0.4pt} %Linea de separacion inferior

\newcommand{\primaria}[1]{
	\textbf{\underline{#1}}
}

\newcommand{\foranea}[1]{
	\textbf{\textsl{#1}}
}

\newcommand{\primyfor}[1]{
	\underline{\foranea{#1}}
}

\makeatletter
\newcommand\subsubsubsection{\@startsection {paragraph}{1}{\z@}%
                                   {-3.5ex \@plus -1ex \@minus -.2ex}%
                                   {1.5ex \@plus.2ex}%
                                   {\normalfont\bfseries}}
                       
                                
                                 
\newcommand\subsubsubsubsection{\@startsection {subparagraph}{1}{\z@}%
                                   {-3.5ex \@plus -1ex \@minus -.2ex}%
                                   {1.5ex \@plus.2ex}%
                                  
                                   {\normalfont\bfseries}}


\makeatother
 

\begin{document}
\title{Investigación de Operaciones I \\ \begin{Large}Tarea 3\end{Large}} 
\author{Victor Gonzalez (2.773.029-9)
\and Cesar Muñoz (2.973.053-0)}
\date{\today}
\maketitle


\section{Maximizando el lucro}

\begin{center}
\begin{tabular}{ccc|ccc|ccc|ccc|ccc|ccc|ccc}
\multicolumn{3}{c|}{} & \multicolumn{3}{c}{Viña del Mar} & \multicolumn{3}{c}{Santiago} & \multicolumn{3}{c}{Iquique} & \multicolumn{3}{c|}{Temuco} & \multicolumn{3}{c}{Oferta(miles productos)}\\
\cline{1-15}
\multicolumn{3}{c|}{} & \multicolumn{3}{r|}{8} & \multicolumn{3}{r|}{5} & \multicolumn{3}{r|}{8} & \multicolumn{3}{r|}{3} \\
\multicolumn{3}{c|}{Juanín} & \multicolumn{3}{c|}{} & \multicolumn{3}{c|}{} & \multicolumn{3}{c|}{} & \multicolumn{3}{c|}{}& \multicolumn{3}{c}{400}\\
\multicolumn{3}{c|}{} & \multicolumn{3}{c|}{} & \multicolumn{3}{c|}{} & \multicolumn{3}{c|}{} & \multicolumn{3}{c|}{}\\
\cline{1-15}
\multicolumn{3}{c|}{} & \multicolumn{3}{r|}{4} & \multicolumn{3}{r|}{6} & \multicolumn{3}{r|}{2} & \multicolumn{3}{r|}{8} \\
\multicolumn{3}{c|}{Tulio} & \multicolumn{3}{c|}{} & \multicolumn{3}{c|}{} & \multicolumn{3}{c|}{} & \multicolumn{3}{c|}{}& \multicolumn{3}{c}{1200}\\
\multicolumn{3}{c|}{} & \multicolumn{3}{c|}{} & \multicolumn{3}{c|}{} & \multicolumn{3}{c|}{} & \multicolumn{3}{c|}{}\\
\cline{1-15}
\multicolumn{3}{c|}{} & \multicolumn{3}{r|}{10} & \multicolumn{3}{r|}{1} & \multicolumn{3}{r|}{3} & \multicolumn{3}{r|}{2} \\
\multicolumn{3}{c|}{Bodoque} & \multicolumn{3}{c|}{} & \multicolumn{3}{c|}{} & \multicolumn{3}{c|}{} & \multicolumn{3}{c|}{}& \multicolumn{3}{c}{500}\\
\multicolumn{3}{c|}{} & \multicolumn{3}{c|}{} & \multicolumn{3}{c|}{} & \multicolumn{3}{c|}{} & \multicolumn{3}{c|}{}\\
\cline{1-15}
\multicolumn{3}{c|}{} & \multicolumn{3}{r|}{7} & \multicolumn{3}{r|}{7} & \multicolumn{3}{r|}{4} & \multicolumn{3}{r|}{8} \\
\multicolumn{3}{c|}{Policarpo} & \multicolumn{3}{c|}{} & \multicolumn{3}{c|}{} & \multicolumn{3}{c|}{} & \multicolumn{3}{c|}{}& \multicolumn{3}{c}{100}\\
\multicolumn{3}{c|}{} & \multicolumn{3}{c|}{} & \multicolumn{3}{c|}{} & \multicolumn{3}{c|}{} & \multicolumn{3}{c|}{}\\
\cline{1-15}
\multicolumn{3}{c|}{} & \multicolumn{3}{r|}{3} & \multicolumn{3}{r|}{4} & \multicolumn{3}{r|}{6} & \multicolumn{3}{r|}{4} \\
\multicolumn{3}{c|}{Huachimingo} & \multicolumn{3}{c|}{} & \multicolumn{3}{c|}{} & \multicolumn{3}{c|}{} & \multicolumn{3}{c|}{}& \multicolumn{3}{c}{750}\\
\multicolumn{3}{c|}{} & \multicolumn{3}{c|}{} & \multicolumn{3}{c|}{} & \multicolumn{3}{c|}{} & \multicolumn{3}{c|}{}\\
\cline{1-15}
\multicolumn{3}{c|}{} & \multicolumn{3}{r|}{9} & \multicolumn{3}{r|}{9} & \multicolumn{3}{r|}{6} & \multicolumn{3}{r|}{5} \\
\multicolumn{3}{c|}{Mario Hugo} & \multicolumn{3}{c|}{} & \multicolumn{3}{c|}{} & \multicolumn{3}{c|}{} & \multicolumn{3}{c|}{}& \multicolumn{3}{c}{800}\\
\multicolumn{3}{c|}{} & \multicolumn{3}{c|}{} & \multicolumn{3}{c|}{} & \multicolumn{3}{c|}{} & \multicolumn{3}{c|}{}\\
\cline{1-15}
\multicolumn{3}{c|}{} & \multicolumn{3}{r|}{4} & \multicolumn{3}{r|}{6} & \multicolumn{3}{r|}{8} & \multicolumn{3}{r|}{7} \\
\multicolumn{3}{c|}{Guaripolo} & \multicolumn{3}{c|}{} & \multicolumn{3}{c|}{} & \multicolumn{3}{c|}{} & \multicolumn{3}{c|}{}& \multicolumn{3}{c}{650}\\
\multicolumn{3}{c|}{} & \multicolumn{3}{c|}{} & \multicolumn{3}{c|}{} & \multicolumn{3}{c|}{} & \multicolumn{3}{c|}{}\\
\cline{1-15}
\multicolumn{3}{c|}{Demanda(millones de pesos)} & \multicolumn{3}{c|}{4,4} & \multicolumn{3}{c|}{4} & \multicolumn{3}{c|}{5,2}& \multicolumn{3}{c|}{4} & \multicolumn{3}{c}{}\\

\end{tabular}
\end{center}
Una vez que se a realizado el metodo de la esquina noroeste, resulta el siguiente tableau:

\begin{center}
\begin{tabular}{ccc|ccc|ccc|ccc|ccc|ccc|ccc}
\multicolumn{3}{c|}{} & \multicolumn{3}{c}{Viña del Mar} & \multicolumn{3}{c}{Santiago} & \multicolumn{3}{c}{Iquique} & \multicolumn{3}{c|}{Temuco} & \multicolumn{3}{c}{Oferta(miles productos)}\\
\cline{1-15}
\multicolumn{3}{c|}{} & \multicolumn{3}{r|}{8} & \multicolumn{3}{r|}{5} & \multicolumn{3}{r|}{8} & \multicolumn{3}{r|}{3} \\
\multicolumn{3}{c|}{Juanín} & \multicolumn{3}{c|}{400} & \multicolumn{3}{c|}{/} & \multicolumn{3}{c|}{/} & \multicolumn{3}{c|}{/}& \multicolumn{3}{c}{400}\\
\multicolumn{3}{c|}{} & \multicolumn{3}{c|}{} & \multicolumn{3}{c|}{} & \multicolumn{3}{c|}{} & \multicolumn{3}{c|}{}\\
\cline{1-15}
\multicolumn{3}{c|}{} & \multicolumn{3}{r|}{4} & \multicolumn{3}{r|}{6} & \multicolumn{3}{r|}{2} & \multicolumn{3}{r|}{8} \\
\multicolumn{3}{c|}{Tulio} & \multicolumn{3}{c|}{300} & \multicolumn{3}{c|}{670} & \multicolumn{3}{c|}{230} & \multicolumn{3}{c|}{/}& \multicolumn{3}{c}{1200}\\
\multicolumn{3}{c|}{} & \multicolumn{3}{c|}{} & \multicolumn{3}{c|}{} & \multicolumn{3}{c|}{} & \multicolumn{3}{c|}{}\\
\cline{1-15}
\multicolumn{3}{c|}{} & \multicolumn{3}{r|}{10} & \multicolumn{3}{r|}{1} & \multicolumn{3}{r|}{3} & \multicolumn{3}{r|}{2} \\
\multicolumn{3}{c|}{Bodoque} & \multicolumn{3}{c|}{/} & \multicolumn{3}{c|}{/} & \multicolumn{3}{c|}{500} & \multicolumn{3}{c|}{/}& \multicolumn{3}{c}{500}\\
\multicolumn{3}{c|}{} & \multicolumn{3}{c|}{} & \multicolumn{3}{c|}{} & \multicolumn{3}{c|}{} & \multicolumn{3}{c|}{}\\
\cline{1-15}
\multicolumn{3}{c|}{} & \multicolumn{3}{r|}{7} & \multicolumn{3}{r|}{7} & \multicolumn{3}{r|}{4} & \multicolumn{3}{r|}{8} \\
\multicolumn{3}{c|}{Policarpo} & \multicolumn{3}{c|}{/} & \multicolumn{3}{c|}{/} & \multicolumn{3}{c|}{100} & \multicolumn{3}{c|}{/}& \multicolumn{3}{c}{100}\\
\multicolumn{3}{c|}{} & \multicolumn{3}{c|}{} & \multicolumn{3}{c|}{} & \multicolumn{3}{c|}{} & \multicolumn{3}{c|}{}\\
\cline{1-15}
\multicolumn{3}{c|}{} & \multicolumn{3}{r|}{3} & \multicolumn{3}{r|}{4} & \multicolumn{3}{r|}{6} & \multicolumn{3}{r|}{4} \\
\multicolumn{3}{c|}{Huachimingo} & \multicolumn{3}{c|}{/} & \multicolumn{3}{c|}{/} & \multicolumn{3}{c|}{474} & \multicolumn{3}{c|}{276}& \multicolumn{3}{c}{750}\\
\multicolumn{3}{c|}{} & \multicolumn{3}{c|}{} & \multicolumn{3}{c|}{} & \multicolumn{3}{c|}{} & \multicolumn{3}{c|}{}\\
\cline{1-15}
\multicolumn{3}{c|}{} & \multicolumn{3}{r|}{9} & \multicolumn{3}{r|}{9} & \multicolumn{3}{r|}{6} & \multicolumn{3}{r|}{5} \\
\multicolumn{3}{c|}{Mario Hugo} & \multicolumn{3}{c|}{/} & \multicolumn{3}{c|}{/} & \multicolumn{3}{c|}{/} & \multicolumn{3}{c|}{580}& \multicolumn{3}{c}{800}\\
\multicolumn{3}{c|}{} & \multicolumn{3}{c|}{} & \multicolumn{3}{c|}{} & \multicolumn{3}{c|}{} & \multicolumn{3}{c|}{}\\
\cline{1-15}
\multicolumn{3}{c|}{} & \multicolumn{3}{r|}{4} & \multicolumn{3}{r|}{6} & \multicolumn{3}{r|}{8} & \multicolumn{3}{r|}{7} \\
\multicolumn{3}{c|}{Guaripolo} & \multicolumn{3}{c|}{/} & \multicolumn{3}{c|}{/} & \multicolumn{3}{c|}{/} & \multicolumn{3}{c|}{/}& \multicolumn{3}{c}{650}\\
\multicolumn{3}{c|}{} & \multicolumn{3}{c|}{} & \multicolumn{3}{c|}{} & \multicolumn{3}{c|}{} & \multicolumn{3}{c|}{}\\
\cline{1-15}
\multicolumn{3}{c|}{Demanda(millones de pesos)} & \multicolumn{3}{c|}{4,4} & \multicolumn{3}{c|}{4} & \multicolumn{3}{c|}{5,2}& \multicolumn{3}{c|}{4} & \multicolumn{3}{c}{}\\
\end{tabular}
\end{center}

Luego de aplicar el método de la esquina noroeste (transporte no óptimo), se puede decir que solo se usarán 6 proveedores (Juanín, Tulio, Bodoque, Policarpo, Huachimingo y Mario Hugo), por lo que queda afuera Guaripolo.
\\ 

Ahora, aplicando el método de Vogel (para obtener una solución optimizada) el primer tableau resulta:
\begin{center}
\begin{tabular}{ccc|ccc|ccc|ccc|ccc|ccc|ccc}
\multicolumn{3}{c|}{} & \multicolumn{3}{c}{Viña del Mar} & \multicolumn{3}{c}{Santiago} & \multicolumn{3}{c}{Iquique} & \multicolumn{3}{c|}{Temuco} & \multicolumn{3}{c}{Oferta(miles productos)}\\
\cline{1-15}
\multicolumn{3}{c|}{} & \multicolumn{3}{r|}{8} & \multicolumn{3}{r|}{5} & \multicolumn{3}{r|}{8} & \multicolumn{3}{r|}{3} \\
\multicolumn{3}{c|}{Juanín} & \multicolumn{3}{c|}{/} & \multicolumn{3}{c|}{/} & \multicolumn{3}{c|}{400} & \multicolumn{3}{c|}{/}& \multicolumn{3}{c}{400}\\
\multicolumn{3}{c|}{} & \multicolumn{3}{c|}{} & \multicolumn{3}{c|}{} & \multicolumn{3}{c|}{} & \multicolumn{3}{c|}{}\\
\cline{1-15}
\multicolumn{3}{c|}{} & \multicolumn{3}{r|}{4} & \multicolumn{3}{r|}{6} & \multicolumn{3}{r|}{2} & \multicolumn{3}{r|}{8} \\
\multicolumn{3}{c|}{Tulio} & \multicolumn{3}{c|}{/} & \multicolumn{3}{c|}{/} & \multicolumn{3}{c|}{/} & \multicolumn{3}{c|}{500}& \multicolumn{3}{c}{1200}\\
\multicolumn{3}{c|}{} & \multicolumn{3}{c|}{} & \multicolumn{3}{c|}{} & \multicolumn{3}{c|}{} & \multicolumn{3}{c|}{}\\
\cline{1-15}
\multicolumn{3}{c|}{} & \multicolumn{3}{r|}{10} & \multicolumn{3}{r|}{1} & \multicolumn{3}{r|}{3} & \multicolumn{3}{r|}{2} \\
\multicolumn{3}{c|}{Bodoque} & \multicolumn{3}{c|}{440} & \multicolumn{3}{c|}{/} & \multicolumn{3}{c|}{/} & \multicolumn{3}{c|}{0}& \multicolumn{3}{c}{500}\\
\multicolumn{3}{c|}{} & \multicolumn{3}{c|}{} & \multicolumn{3}{c|}{} & \multicolumn{3}{c|}{} & \multicolumn{3}{c|}{}\\
\cline{1-15}
\multicolumn{3}{c|}{} & \multicolumn{3}{r|}{7} & \multicolumn{3}{r|}{7} & \multicolumn{3}{r|}{4} & \multicolumn{3}{r|}{8} \\
\multicolumn{3}{c|}{Policarpo} & \multicolumn{3}{c|}{/} & \multicolumn{3}{c|}{/} & \multicolumn{3}{c|}{/} & \multicolumn{3}{c|}{0}& \multicolumn{3}{c}{100}\\
\multicolumn{3}{c|}{} & \multicolumn{3}{c|}{} & \multicolumn{3}{c|}{} & \multicolumn{3}{c|}{} & \multicolumn{3}{c|}{}\\
\cline{1-15}
\multicolumn{3}{c|}{} & \multicolumn{3}{r|}{3} & \multicolumn{3}{r|}{4} & \multicolumn{3}{r|}{6} & \multicolumn{3}{r|}{4} \\
\multicolumn{3}{c|}{Huachimingo} & \multicolumn{3}{c|}{/} & \multicolumn{3}{c|}{/} & \multicolumn{3}{c|}{/} & \multicolumn{3}{c|}{0}& \multicolumn{3}{c}{750}\\
\multicolumn{3}{c|}{} & \multicolumn{3}{c|}{} & \multicolumn{3}{c|}{} & \multicolumn{3}{c|}{} & \multicolumn{3}{c|}{}\\
\cline{1-15}
\multicolumn{3}{c|}{} & \multicolumn{3}{r|}{9} & \multicolumn{3}{r|}{9} & \multicolumn{3}{r|}{6} & \multicolumn{3}{r|}{5} \\
\multicolumn{3}{c|}{Mario Hugo} & \multicolumn{3}{c|}{/} & \multicolumn{3}{c|}{445} & \multicolumn{3}{c|}{/} & \multicolumn{3}{c|}{0}& \multicolumn{3}{c}{800}\\
\multicolumn{3}{c|}{} & \multicolumn{3}{c|}{} & \multicolumn{3}{c|}{} & \multicolumn{3}{c|}{} & \multicolumn{3}{c|}{}\\
\cline{1-15}
\multicolumn{3}{c|}{} & \multicolumn{3}{r|}{4} & \multicolumn{3}{r|}{6} & \multicolumn{3}{r|}{8} & \multicolumn{3}{r|}{7} \\
\multicolumn{3}{c|}{Guaripolo} & \multicolumn{3}{c|}{/} & \multicolumn{3}{c|}{/} & \multicolumn{3}{c|}{250} & \multicolumn{3}{c|}{0}& \multicolumn{3}{c}{650}\\
\multicolumn{3}{c|}{} & \multicolumn{3}{c|}{} & \multicolumn{3}{c|}{} & \multicolumn{3}{c|}{} & \multicolumn{3}{c|}{}\\
\cline{1-15}
\multicolumn{3}{c|}{Demanda(millones de pesos)} & \multicolumn{3}{c|}{4,4} & \multicolumn{3}{c|}{4} & \multicolumn{3}{c|}{5,2}& \multicolumn{3}{c|}{4} & \multicolumn{3}{c}{}\\

\end{tabular}
\end{center}

Veificando m+n-1, donde m es la cantidad de filas y n la cantidad de columnas, resulta: 7+4-1=10. Lo que confirma la cantidad de asignaciones
que tiene el tableau, donde 5 de ellas son diferentes de 0 y las restantes son 0(para poder resolver el tableau completamente).
\\

Luego, tenemos: \\

$u_{1}+v_{3}=8$ \\
$u_{2}+v_{4}=8$ \\
$u_{3}+v_{1}=10$ \\
$u_{3}+v_{4}=2$ \\
$u_{4}+v_{4}=8$ \\
$u_{5}+v_{4}=4$ \\
$u_{6}+v_{2}=9$ \\
$u_{6}+v_{4}=5$ \\
$u_{7}+v_{3}=8$ \\
$u_{7}+v_{4}=7$ \\

Sabiendo que $u_{1}=0$, vamos sacando los valores de todas las variables: \\

$u_{2}=1$ \\
$u_{3}=-5$ \\
$u_{4}=1$ \\
$u_{5}=-3$ \\
$u_{6}=-2$ \\
$u_{7}=0$ \\
$v_{1}=15$ \\
$v_{2}=11$ \\
$v_{3}=8$ \\
$v_{4}=7$ \\

Ahora se calcula el valor de $e_{ij}=X-u_{i}-v_{j}$. \\

$e_{11}=8-0-15=-7$ \\
$e_{12}=5-0-11=-6$ \\
$e_{13}=8-0-8= 0$\\
$e_{14}=3-0-7=-4$ \\
$e_{22}=6-1-15=-10$ \\
$e_{23}=2-1-8=-7$ \\
$e_{24}=8-1-7=0$ \\
$e_{33}=3+5-8=0$ \\
$e_{34}=2+5-7=0$ \\
$e_{44}=8-1-7=0$ \\
$e_{51}=3+3-15=-9$ \\
$e_{52}=4+3-11=-4$ \\
$e_{53}=6+3-8=1$ \\
$e_{54}=4+3-7=0$ \\
$e_{61}=9+2-15=-4$ \\
$e_{62}=9+2-11=0$ \\
$e_{63}=6+2-8=0$ \\
$e_{64}=5+2-7=0$ \\
$e_{71}=4-0-15=-11$ \\
$e_{72}=6-0-11=-5$ \\
$e_{73}=8-0-8=0$ \\
$e_{74}=7-0-7$ \\

Como resulto que en $e_{53}$fue igual a 1 (debiendo dar todos los resultados de $e_{ij}\leq0$ para obtener la solución óptima, ya que se esta máximizando la ganacia),
se debe agregar un factor $\alpha$ para alcanzar la sólución óptima. Es por ello que resulta el siguiente tableau: 
\newline

\begin{center}
\begin{tabular}{ccc|ccc|ccc|ccc|ccc|ccc|ccc}
\multicolumn{3}{c|}{} & \multicolumn{3}{c}{Viña del Mar} & \multicolumn{3}{c}{Santiago} & \multicolumn{3}{c}{Iquique} & \multicolumn{3}{c|}{Temuco} & \multicolumn{3}{c}{Oferta(miles productos)}\\
\cline{1-15}
\multicolumn{3}{c|}{} & \multicolumn{3}{r|}{8} & \multicolumn{3}{r|}{5} & \multicolumn{3}{r|}{8} & \multicolumn{3}{r|}{3} \\
\multicolumn{3}{c|}{Juanín} & \multicolumn{3}{c|}{/} & \multicolumn{3}{c|}{/} & \multicolumn{3}{c|}{400} & \multicolumn{3}{c|}{/}& \multicolumn{3}{c}{400}\\
\multicolumn{3}{c|}{} & \multicolumn{3}{c|}{} & \multicolumn{3}{c|}{} & \multicolumn{3}{c|}{} & \multicolumn{3}{c|}{}\\
\cline{1-15}
\multicolumn{3}{c|}{} & \multicolumn{3}{r|}{4} & \multicolumn{3}{r|}{6} & \multicolumn{3}{r|}{2} & \multicolumn{3}{r|}{8} \\
\multicolumn{3}{c|}{Tulio} & \multicolumn{3}{c|}{/} & \multicolumn{3}{c|}{/} & \multicolumn{3}{c|}{/} & \multicolumn{3}{c|}{500}& \multicolumn{3}{c}{1200}\\
\multicolumn{3}{c|}{} & \multicolumn{3}{c|}{} & \multicolumn{3}{c|}{} & \multicolumn{3}{c|}{} & \multicolumn{3}{c|}{}\\
\cline{1-15}
\multicolumn{3}{c|}{} & \multicolumn{3}{r|}{10} & \multicolumn{3}{r|}{1} & \multicolumn{3}{r|}{3} & \multicolumn{3}{r|}{2} \\
\multicolumn{3}{c|}{Bodoque} & \multicolumn{3}{c|}{440} & \multicolumn{3}{c|}{/} & \multicolumn{3}{c|}{/} & \multicolumn{3}{c|}{0}& \multicolumn{3}{c}{500}\\
\multicolumn{3}{c|}{} & \multicolumn{3}{c|}{} & \multicolumn{3}{c|}{} & \multicolumn{3}{c|}{} & \multicolumn{3}{c|}{}\\
\cline{1-15}
\multicolumn{3}{c|}{} & \multicolumn{3}{r|}{7} & \multicolumn{3}{r|}{7} & \multicolumn{3}{r|}{4} & \multicolumn{3}{r|}{8} \\
\multicolumn{3}{c|}{Policarpo} & \multicolumn{3}{c|}{/} & \multicolumn{3}{c|}{/} & \multicolumn{3}{c|}{/} & \multicolumn{3}{c|}{0}& \multicolumn{3}{c}{100}\\
\multicolumn{3}{c|}{} & \multicolumn{3}{c|}{} & \multicolumn{3}{c|}{} & \multicolumn{3}{c|}{} & \multicolumn{3}{c|}{}\\
\cline{1-15}
\multicolumn{3}{c|}{} & \multicolumn{3}{r|}{3} & \multicolumn{3}{r|}{4} & \multicolumn{3}{r|}{6} & \multicolumn{3}{r|}{4} \\
\multicolumn{3}{c|}{Huachimingo} & \multicolumn{3}{c|}{/} & \multicolumn{3}{c|}{/} & \multicolumn{3}{c|}{$\alpha$} & \multicolumn{3}{c|}{$0-\alpha$}& \multicolumn{3}{c}{750}\\
\multicolumn{3}{c|}{} & \multicolumn{3}{c|}{} & \multicolumn{3}{c|}{} & \multicolumn{3}{c|}{} & \multicolumn{3}{c|}{}\\
\cline{1-15}
\multicolumn{3}{c|}{} & \multicolumn{3}{r|}{9} & \multicolumn{3}{r|}{9} & \multicolumn{3}{r|}{6} & \multicolumn{3}{r|}{5} \\
\multicolumn{3}{c|}{Mario Hugo} & \multicolumn{3}{c|}{/} & \multicolumn{3}{c|}{445} & \multicolumn{3}{c|}{/} & \multicolumn{3}{c|}{0}& \multicolumn{3}{c}{800}\\
\multicolumn{3}{c|}{} & \multicolumn{3}{c|}{} & \multicolumn{3}{c|}{} & \multicolumn{3}{c|}{} & \multicolumn{3}{c|}{}\\
\cline{1-15}
\multicolumn{3}{c|}{} & \multicolumn{3}{r|}{4} & \multicolumn{3}{r|}{6} & \multicolumn{3}{r|}{8} & \multicolumn{3}{r|}{7} \\
\multicolumn{3}{c|}{Guaripolo} & \multicolumn{3}{c|}{/} & \multicolumn{3}{c|}{/} & \multicolumn{3}{c|}{$250-\alpha$} & \multicolumn{3}{c|}{$0+\alpha$}& \multicolumn{3}{c}{650}\\
\multicolumn{3}{c|}{} & \multicolumn{3}{c|}{} & \multicolumn{3}{c|}{} & \multicolumn{3}{c|}{} & \multicolumn{3}{c|}{}\\
\cline{1-15}
\multicolumn{3}{c|}{Demanda(millones de pesos)} & \multicolumn{3}{c|}{4,4} & \multicolumn{3}{c|}{4} & \multicolumn{3}{c|}{5,2}& \multicolumn{3}{c|}{4} & \multicolumn{3}{c}{}\\

\end{tabular}
\end{center}
$0-alpha\leq 0$ ó $250-alpha\leq 0$. Luego, $\alpha = 0$. \\

\begin{center}
\begin{tabular}{ccc|ccc|ccc|ccc|ccc|ccc|ccc}
\multicolumn{3}{c|}{} & \multicolumn{3}{c}{Viña del Mar} & \multicolumn{3}{c}{Santiago} & \multicolumn{3}{c}{Iquique} & \multicolumn{3}{c|}{Temuco} & \multicolumn{3}{c}{Oferta(miles productos)}\\
\cline{1-15}
\multicolumn{3}{c|}{} & \multicolumn{3}{r|}{8} & \multicolumn{3}{r|}{5} & \multicolumn{3}{r|}{8} & \multicolumn{3}{r|}{3} \\
\multicolumn{3}{c|}{Juanín} & \multicolumn{3}{c|}{/} & \multicolumn{3}{c|}{/} & \multicolumn{3}{c|}{400} & \multicolumn{3}{c|}{/}& \multicolumn{3}{c}{400}\\
\multicolumn{3}{c|}{} & \multicolumn{3}{c|}{} & \multicolumn{3}{c|}{} & \multicolumn{3}{c|}{} & \multicolumn{3}{c|}{}\\
\cline{1-15}
\multicolumn{3}{c|}{} & \multicolumn{3}{r|}{4} & \multicolumn{3}{r|}{6} & \multicolumn{3}{r|}{2} & \multicolumn{3}{r|}{8} \\
\multicolumn{3}{c|}{Tulio} & \multicolumn{3}{c|}{/} & \multicolumn{3}{c|}{/} & \multicolumn{3}{c|}{/} & \multicolumn{3}{c|}{500}& \multicolumn{3}{c}{1200}\\
\multicolumn{3}{c|}{} & \multicolumn{3}{c|}{} & \multicolumn{3}{c|}{} & \multicolumn{3}{c|}{} & \multicolumn{3}{c|}{}\\
\cline{1-15}
\multicolumn{3}{c|}{} & \multicolumn{3}{r|}{10} & \multicolumn{3}{r|}{1} & \multicolumn{3}{r|}{3} & \multicolumn{3}{r|}{2} \\
\multicolumn{3}{c|}{Bodoque} & \multicolumn{3}{c|}{440} & \multicolumn{3}{c|}{/} & \multicolumn{3}{c|}{/} & \multicolumn{3}{c|}{0}& \multicolumn{3}{c}{500}\\
\multicolumn{3}{c|}{} & \multicolumn{3}{c|}{} & \multicolumn{3}{c|}{} & \multicolumn{3}{c|}{} & \multicolumn{3}{c|}{}\\
\cline{1-15}
\multicolumn{3}{c|}{} & \multicolumn{3}{r|}{7} & \multicolumn{3}{r|}{7} & \multicolumn{3}{r|}{4} & \multicolumn{3}{r|}{8} \\
\multicolumn{3}{c|}{Policarpo} & \multicolumn{3}{c|}{/} & \multicolumn{3}{c|}{/} & \multicolumn{3}{c|}{/} & \multicolumn{3}{c|}{0}& \multicolumn{3}{c}{100}\\
\multicolumn{3}{c|}{} & \multicolumn{3}{c|}{} & \multicolumn{3}{c|}{} & \multicolumn{3}{c|}{} & \multicolumn{3}{c|}{}\\
\cline{1-15}
\multicolumn{3}{c|}{} & \multicolumn{3}{r|}{3} & \multicolumn{3}{r|}{4} & \multicolumn{3}{r|}{6} & \multicolumn{3}{r|}{4} \\
\multicolumn{3}{c|}{Huachimingo} & \multicolumn{3}{c|}{/} & \multicolumn{3}{c|}{/} & \multicolumn{3}{c|}{0} & \multicolumn{3}{c|}{/}& \multicolumn{3}{c}{750}\\
\multicolumn{3}{c|}{} & \multicolumn{3}{c|}{} & \multicolumn{3}{c|}{} & \multicolumn{3}{c|}{} & \multicolumn{3}{c|}{}\\
\cline{1-15}
\multicolumn{3}{c|}{} & \multicolumn{3}{r|}{9} & \multicolumn{3}{r|}{9} & \multicolumn{3}{r|}{6} & \multicolumn{3}{r|}{5} \\
\multicolumn{3}{c|}{Mario Hugo} & \multicolumn{3}{c|}{/} & \multicolumn{3}{c|}{445} & \multicolumn{3}{c|}{/} & \multicolumn{3}{c|}{0}& \multicolumn{3}{c}{800}\\
\multicolumn{3}{c|}{} & \multicolumn{3}{c|}{} & \multicolumn{3}{c|}{} & \multicolumn{3}{c|}{} & \multicolumn{3}{c|}{}\\
\cline{1-15}
\multicolumn{3}{c|}{} & \multicolumn{3}{r|}{4} & \multicolumn{3}{r|}{6} & \multicolumn{3}{r|}{8} & \multicolumn{3}{r|}{7} \\
\multicolumn{3}{c|}{Guaripolo} & \multicolumn{3}{c|}{/} & \multicolumn{3}{c|}{/} & \multicolumn{3}{c|}{250} & \multicolumn{3}{c|}{0}& \multicolumn{3}{c}{650}\\
\multicolumn{3}{c|}{} & \multicolumn{3}{c|}{} & \multicolumn{3}{c|}{} & \multicolumn{3}{c|}{} & \multicolumn{3}{c|}{}\\
\cline{1-15}
\multicolumn{3}{c|}{Demanda(millones de pesos)} & \multicolumn{3}{c|}{4,4} & \multicolumn{3}{c|}{4} & \multicolumn{3}{c|}{5,2}& \multicolumn{3}{c|}{4} & \multicolumn{3}{c}{}\\

\end{tabular}
\end{center}

Luego, tenemos: \\

$u_{1}+v_{3}=8$ \\
$u_{2}+v_{4}=8$ \\
$u_{3}+v_{1}=10$ \\
$u_{3}+v_{4}=2$ \\
$u_{4}+v_{4}=8$ \\
$u_{5}+v_{3}=6$ \\
$u_{6}+v_{2}=9$ \\
$u_{6}+v_{4}=5$ \\
$u_{7}+v_{3}=8$ \\
$u_{7}+v_{4}=7$ \\

Sabiendo que $u_{1}=0$, vamos sacando los valores de todas las variables: \\

$u_{2}=1$ \\
$u_{3}=-5$ \\
$u_{4}=1$ \\
$u_{5}=-2$ \\
$u_{6}=-2$ \\
$u_{7}=0$ \\
$v_{1}=15$ \\
$v_{2}=11$ \\
$v_{3}=8$ \\
$v_{4}=7$ \\

Ahora se calcula el valor de $e_{ij}=X-u_{i}-v_{j}$. \\

$e_{11}=8-0-15=-7$ \\
$e_{12}=5-0-11=-6$ \\
$e_{13}=8-0-8= 0$\\
$e_{14}=3-0-7=-4$ \\
$e_{22}=6-1-15=-10$ \\
$e_{23}=2-1-8=-7$ \\
$e_{24}=8-1-7=0$ \\
$e_{33}=3+5-8=0$ \\
$e_{34}=2+5-7=0$ \\
$e_{44}=8-1-7=0$ \\
$e_{51}=3+2-15=-10$ \\
$e_{52}=4+2-11=-5$ \\
$e_{53}=6+2-8=0$ \\
$e_{54}=4+2-7=-1$ \\
$e_{61}=9+2-15=-4$ \\
$e_{62}=9+2-11=0$ \\
$e_{63}=6+2-8=0$ \\
$e_{64}=5+2-7=0$ \\
$e_{71}=4-0-15=-11$ \\
$e_{72}=6-0-11=-5$ \\
$e_{73}=8-0-8=0$ \\
$e_{74}=7-0-7$ \\

En base al método Vogel, podemos decir que solo se utilizarán 5 proveedores, los cuales son: Juanín, Tulio, Bodoque, Mario Hugo y Guaripolo. Quedando afuera: Huachimingo y Policarpo.\\
Ganancia= 4400 + 4005 + 3200 + 2000 + 4000 = 17605*1000

Parte 2. \\
Ahora, se quitan los 2 proveedores que no se usarán y se agrega una columna para poder dejar la oferta en 0.
\begin{center}
\begin{tabular}{ccc|ccc|ccc|ccc|ccc|ccc|ccc}
\multicolumn{3}{c|}{} & \multicolumn{3}{c}{Viña del Mar} & \multicolumn{3}{c}{Santiago} & \multicolumn{3}{c}{Iquique} & \multicolumn{3}{c|}{Temuco} & \multicolumn{3}{c|}{Dummy} & \multicolumn{3}{c}{Oferta(miles productos)}\\
\cline{1-15}
\multicolumn{3}{c|}{} & \multicolumn{3}{r|}{8} & \multicolumn{3}{r|}{5} & \multicolumn{3}{r|}{8} & \multicolumn{3}{r|}{3} & \multicolumn{3}{r|}{0}\\
\multicolumn{3}{c|}{Juanín} & \multicolumn{3}{c|}{/} & \multicolumn{3}{c|}{/} & \multicolumn{3}{c|}{400} & \multicolumn{3}{c|}{/} & \multicolumn{3}{c|}{700} & \multicolumn{3}{c}{400}\\
\multicolumn{3}{c|}{} & \multicolumn{3}{c|}{} & \multicolumn{3}{c|}{} & \multicolumn{3}{c|}{} & \multicolumn{3}{c|}{} & \multicolumn{3}{c|}{/}\\
\cline{1-15}
\multicolumn{3}{c|}{} & \multicolumn{3}{r|}{4} & \multicolumn{3}{r|}{6} & \multicolumn{3}{r|}{2} & \multicolumn{3}{r|}{8} & \multicolumn{3}{r|}{0}\\
\multicolumn{3}{c|}{Tulio} & \multicolumn{3}{c|}{/} & \multicolumn{3}{c|}{/} & \multicolumn{3}{c|}{/} & \multicolumn{3}{c|}{500} & \multicolumn{3}{c|}{60} & \multicolumn{3}{c}{1200}\\
\multicolumn{3}{c|}{} & \multicolumn{3}{c|}{} & \multicolumn{3}{c|}{} & \multicolumn{3}{c|}{} & \multicolumn{3}{c|}{} & \multicolumn{3}{c|}{}\\
\cline{1-15}
\multicolumn{3}{c|}{} & \multicolumn{3}{r|}{10} & \multicolumn{3}{r|}{1} & \multicolumn{3}{r|}{3} & \multicolumn{3}{r|}{2} & \multicolumn{3}{r|}{0}\\
\multicolumn{3}{c|}{Bodoque} & \multicolumn{3}{c|}{440} & \multicolumn{3}{c|}{/} & \multicolumn{3}{c|}{/} & \multicolumn{3}{c|}{0} & \multicolumn{3}{c|}{60}& \multicolumn{3}{c}{500}\\
\multicolumn{3}{c|}{} & \multicolumn{3}{c|}{} & \multicolumn{3}{c|}{} & \multicolumn{3}{c|}{} & \multicolumn{3}{c|}{} & \multicolumn{3}{c|}{}\\
\cline{1-15}
\multicolumn{3}{c|}{} & \multicolumn{3}{r|}{9} & \multicolumn{3}{r|}{9} & \multicolumn{3}{r|}{6} & \multicolumn{3}{r|}{5} & \multicolumn{3}{r|}{0} \\
\multicolumn{3}{c|}{Mario Hugo} & \multicolumn{3}{c|}{/} & \multicolumn{3}{c|}{445} & \multicolumn{3}{c|}{/} & \multicolumn{3}{c|}{0}& \multicolumn{3}{c|}{355} & \multicolumn{3}{c}{800}\\
\multicolumn{3}{c|}{} & \multicolumn{3}{c|}{} & \multicolumn{3}{c|}{} & \multicolumn{3}{c|}{} & \multicolumn{3}{c|}{} & \multicolumn{3}{c|}{}\\
\cline{1-15}
\multicolumn{3}{c|}{} & \multicolumn{3}{r|}{4} & \multicolumn{3}{r|}{6} & \multicolumn{3}{r|}{8} & \multicolumn{3}{r|}{7} & \multicolumn{3}{r|}{0}\\
\multicolumn{3}{c|}{Guaripolo} & \multicolumn{3}{c|}{/} & \multicolumn{3}{c|}{/} & \multicolumn{3}{c|}{250} & \multicolumn{3}{c|}{0}  & \multicolumn{3}{c|}{400} & \multicolumn{3}{c}{650}\\
\multicolumn{3}{c|}{} & \multicolumn{3}{c|}{} & \multicolumn{3}{c|}{} & \multicolumn{3}{c|}{} & \multicolumn{3}{c|}{} & \multicolumn{3}{c|}{}\\
\cline{1-15}
\multicolumn{3}{c|}{Demanda(millones de pesos)} & \multicolumn{3}{c|}{4,4} & \multicolumn{3}{c|}{4} & \multicolumn{3}{c|}{5,2}& \multicolumn{3}{c|}{4}& \multicolumn{3}{c}{0} & \multicolumn{3}{c}{}\\

\end{tabular}
\end{center}

Luego, tenemos: \\

$u_{1}+v_{3}=8$ \\
$u_{2}+v_{4}=8$ \\
$u_{2}+v_{5}=0$ \\
$u_{3}+v_{1}=10$ \\
$u_{3}+v_{5}=0$ \\
$u_{4}+v_{2}=9$ \\
$u_{4}+v_{5}=0$ \\
$u_{5}+v_{3}=8$ \\
$u_{5}+v_{5}=0$ \\

Sabiendo que $u_{1}=0$, vamos sacando los valores de todas las variables: \\

$u_{2}=0$ \\
$u_{3}=0$ \\
$u_{4}=0$ \\
$u_{5}=0$ \\
$v_{1}=10$ \\
$v_{2}=9$ \\
$v_{3}=8$ \\
$v_{4}=8$ \\
$v_{5}=0$ \\

Ahora se calcula el valor de $e_{ij}=X-u_{i}-v_{j}$. \\

$e_{11}=8-0-10=-2$ \\
$e_{12}=5-0-9=-4$ \\
$e_{13}=8-0-8= 0$\\
$e_{14}=3-0-8=-5$ \\
$e_{15}=0-0-0=0$ \\
$e_{22}=6-0-9=-3$ \\
$e_{23}=2-0-8=-6$ \\
$e_{24}=8-0-8=0$ \\
$e_{25}=0-0-0=0$ \\
$e_{33}=3-0-8=-5$ \\
$e_{34}=2-0--8=-6$ \\
$e_{35}=0-0-0=0$ \\
$e_{44}=5-0-8=-3$ \\
$e_{45}=0-0-0=0$ \\
$e_{55}=0-0-0=0$ \\

Esto nos indica que es la solución óptima, ya que todos los valores de $e_{ij}$ son $\leq$ 0.



\subsection{¿Cuál es la mínima cantidad de proveedores requerida para cumplir con las ganancias pedido y quiénes serían?}
\subsubsection{Solución no óptima}
\subsubsection{Solución optimizada}

\subsection{Solución con todos los vendedores}
%\input{./src/tarea2_parte1_costo.ltx}


\section{NutraFat Alimentos}
\subsection{Condiciones de tiempo}
\subsubsection{¿Cual es el recorrido óptimo que debe hacer el chofer?}
Para resolver este problema, se hizo un matriz con todas los tiempos entre los destinos, pero dada la complejidad por la cantidad de conexiones, se recurrió al uso de \textit{lp\_solve}.

En primera instancia se programó el siguiente código:

\begin{verbatim}
/* Función objetivo: minimizar tiempos */
min: 999999 x11 + 999999 x22 + 999999 x33 + 999999 x44 + 999999 x55 + 999999 x66
 + 999999 x77 + 999999 x88 + 999999 x99 + 999999 x14 + 999999 x41 + 999999 x18 +
999999 x81 + 999999 x19 + 999999 x91 + 999999 x25 + 999999 x52 + 999999 x26 +
999999 x62 + 999999 x28 + 999999 x82 + 999999 x36 + 999999 x63 + 999999 x37 +
999999 x73 + 999999 x39 + 999999 x93 + 999999 x46 + 999999 x64 + 999999 x48 +
999999 x84 + 999999 x59 + 999999 x95 + 999999 x67 + 999999 x76 + 999999 x68 +
999999 x86 + 10 x12 + 10 x21 + 24 x13 + 24 x31 + 30 x15 + 30 x51 + 15 x16 +
15 x61 + 12 x17 + 12 x71 + 35 x23 + 35 x32 + 17 x24 + 17 x42 + 27 x34 + 27 x43
+ 25 x35 + 25 x53 + 20 x38 + 20 x83 + 12 x45 + 12 x54 + 18 x47 + 18 x74 + 15 x49
+ 15 x94 + 20 x56 + 20 x65 + 24 x57 + 24 x75 + 40 x58 + 40 x85 + 21 x69 + 21 x96
+ 5 x78 + 21 x79 + 15 x87 + 23 x89 + 15 x97 + 7 x98 + 20 x29 + 20 x92
+ 8 x27 + 8 x72; 

/* Método hungaro */
/* Máximo una asignación por fila */
x11 + x12 + x13 + x14 + x15 + x16 + x17 + x18 + x19 = 1;
x21 + x22 + x23 + x24 + x25 + x26 + x27 + x28 + x29 = 1;
x31 + x32 + x33 + x34 + x35 + x36 + x37 + x38 + x39 = 1;
x41 + x42 + x43 + x44 + x45 + x46 + x47 + x48 + x49 = 1;
x51 + x52 + x53 + x54 + x55 + x56 + x57 + x58 + x59 = 1;
x61 + x62 + x63 + x64 + x65 + x66 + x67 + x68 + x69 = 1;
x71 + x72 + x73 + x74 + x75 + x76 + x77 + x78 + x79 = 1;
x81 + x82 + x83 + x84 + x85 + x86 + x87 + x88 + x89 = 1;
x91 + x92 + x93 + x94 + x95 + x96 + x97 + x98 + x99 = 1;

/* Máximo una asignación por columna */
x11 + x21 + x31 + x41 + x51 + x61 + x71 + x81 + x91 = 1;
x12 + x22 + x32 + x42 + x52 + x62 + x72 + x82 + x92 = 1;
x13 + x23 + x33 + x43 + x53 + x63 + x73 + x83 + x93 = 1;
x14 + x24 + x34 + x44 + x54 + x64 + x74 + x84 + x94 = 1;
x15 + x25 + x35 + x45 + x55 + x65 + x75 + x85 + x95 = 1;
x16 + x26 + x36 + x46 + x56 + x66 + x76 + x86 + x96 = 1;
x17 + x27 + x37 + x47 + x57 + x67 + x77 + x87 + x97 = 1;
x18 + x28 + x38 + x48 + x58 + x68 + x78 + x88 + x98 = 1;
x19 + x29 + x39 + x49 + x59 + x69 + x79 + x89 + x99 = 1;
\end{verbatim}

\{1,2,3,4,5,6,7,8,9\} = \{HQ, AL, DAN, JP, USM, NIC, PUCV, YON, NIL\}

x12 representa el camino entre HQ y AL.

\textit{\textbf{Nota:} no es necesario indicar la naturaleza de las variables dada las restricciones del problema.}
\\

El resultado de esto nos indica que la ruta a seguir por el repartidor es:
$HQ \to  NIC \to HQ$ y $AL \to PUCV \to AL$ y $DAN \to USM \to JP \to NIL \to YON \to DAN$. Con un tiempo total del viaje
de 125 minutos.

Podemos identificar inmediatamente que hay un loop entre lo que es HQ y NIC, y ademas entre AL y PUCV, por lo que agregaremos nuevas restricciones, para quebrar esos loops. En primera instancia se prueba una ramificación del problema, tal que: \verb+x16 = 0;+ y se modifica la F.O. a $... + 999999 x_{16} + ...$. Por otro lado se prueba también con la alternativa de: \verb+x61 = 0;+ y se modifica la F.O. a $... + 999999 x_{61} + ...$. Y de manera análoga se realiza para $x_{27}$ y $x_{72}$.
\\

Cada una de esas ramificaciones generará nuevos loops, pero resumiremos la información obtenida, solo para que se entienda el caso general.

Al romper el camino entre $HQ \to NIC$, el tiempo de viaje queda en 131. Al romper entre $NIC \to HQ$, el tiempo de viaje queda en 127. Al romper entre $AL \to PUCV$, el tiempo de viaje queda en 133. Y finalmente al romper entre $PUCV \to AL$, el tiempo de viaje queda en 131.

Claramente nuestra opción óptima se obtendrá al seguir la ramificación cuando se rompe $NIC \to HQ$. Por lo cual se sigue buscando el óptimo por esa rama.
Lamentablemente, esto se vuelve a repetir un par de veces más al seguir generando resultados, pero el resultado final se obtiene al romper $NIC \to HQ$, $JP \to USM$, $AL \to PUCV$ y $YON \to DAN$ (en ese orden), el cual entrega un tiempo óptimo de viaje de 136 minutos mediante la siguiente ruta:

\begin{center}$HQ \to NIC \to NIL \to YON \to DAN \to USM \to JP \to PUCV \to AL \to HQ$\end{center}

\subsubsection{¿Tiene otras posibilidades de rutas?}
La posibilidad de ruta que tenía este problema, era al encontrarse con las ramificaciónes de $AL \to PUCV$ y $PUCV \to AL$, ya que al romper estas rutas, ambas daban un recorrido de 135 minutos, pero al ramificar la ruta sin el camino $AL \to PUCV$, se encontró el óptimo (luego de quebrar tambien $DAN \to YON$). Y por la otra rama, cuando se rompe $PUCV \to AL$, se generaron 4 ramas más al romper $JP \to NIL$, $NIL \to NIC$, $NIC \to USM$ y $USM \to JP$, cada una de las cuales generaba una solución no óptima con un tiempo de recorrido de 137 minutos.

Por lo tanto, no existen rutas alternativas óptimas.

\subsubsection{¿Grafo de recorrido?}
A partir de lo obtenido, el grafo óptimo del recorrido del repartidor de NutraFat, es:

\begin{center}$HQ \to NIC \to NIL \to YON \to DAN \to USM \to JP \to PUCV \to AL \to HQ$\end{center}

\subsection{Condiciones de costo}
\subsubsection{¿Cual es la ruta más económica para el chofer?}
Para resolver este problema, se utilizó el siguiente código:
\begin{verbatim}
/* Función objetivo: minimizar costos*/
min: 999999 x11 + 999999 x22 + 999999 x33 + 999999 x44 + 999999 x55 + 999999 x66
+ 999999 x14 + 999999 x41 + 999999 x25 + 999999 x52 + 999999 x26 + 999999 x62 +
999999 x36 + 999999 x63 + 999999 x46 + 999999 x64 + 12 x12 + 12 x21 + 16 x13 +
16 x31 + 25 x15 + 25 x51 + 10 x16 + 10 x61 +  23 x23 + 23 x32 + 19 x24 + 19 x42
+ 27 x34 + 27 x43 + 22 x35 + 22 x53 + 20 x45 + 20 x54 + 14 x56 + 14 x65;

/* Método hungaro */
/* Máximo una asignación por fila */
x11 + x12 + x13 + x14 + x15 + x16 = 1;
x21 + x22 + x23 + x24 + x25 + x26 = 1;
x31 + x32 + x33 + x34 + x35 + x36 = 1;
x41 + x42 + x43 + x44 + x45 + x46 = 1;
x51 + x52 + x53 + x54 + x55 + x56 = 1;
x61 + x62 + x63 + x64 + x65 + x66 = 1;

/* Máximo una asignación por columna */
x11 + x21 + x31 + x41 + x51 + x61 = 1;
x12 + x22 + x32 + x42 + x52 + x62 = 1;
x13 + x23 + x33 + x43 + x53 + x63 = 1;
x14 + x24 + x34 + x44 + x54 + x64 = 1;
x15 + x25 + x35 + x45 + x55 + x65 = 1;
x16 + x26 + x36 + x46 + x56 + x66 = 1;
\end{verbatim}

Al hacer funcionar este código, recibimos por resultado la siguiente ruta: $HQ \to DAN \to HQ$, $AL \to JP \to AL$ y $USM \to NIC \to USM$.

Claramente estamos frente a muchos loops, y al hacer las ramificaciones correspondientes para resolver este problema, se generan 6 ramas en primera instancia, de las cuales solo 4 son óptimas, con un costo para el repartidor de 100 (las otras son de 102).

Estas 4 ramas, generan soluciones nuevamente con loops, los cuales deben ser quebrados nuevamente, por lo que se generan en total 8 nuevas ramas, de las cuales solo 6 son óptimas con una solución de costo de 102. Aquí se generan muchas rutas alternativas, pero para hacer esto más rápido, pondremos el caso cuando rompemos la ruta entre $HQ \to DAN$ y $AL \to JP$ (en ese orden).
\newpage
Para encontrar la solución, tuvimos que ingresar en el código las siguientes modificaciones: en las restricciones se agregó \verb+x13 = 0;+, y \verb+x24 = 0;+. En la función objetivo, se modificó $... + 999999 x_{13} + 999999 x_{24} + ...$.

Con estos cambios, se pudo obtener \textbf{un} camino óptimo (de los tantos), el cual es:
\begin{center}$HQ \to NIC \to USM \to JP \to AL \to DAN \to HQ$\end{center}

\bibliographystyle{alpha}
\bibliography{bibbase}

% referencias

\end{document}
