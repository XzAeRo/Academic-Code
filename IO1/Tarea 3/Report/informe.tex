\documentclass[12pt,letterpaper]{article}
\usepackage[utf8]{inputenx} %Codificacion del texto (ISO Latin1 encoding)

\usepackage{fancyhdr} %Permite acomodar a tu gusto la parte de arriba y
% abajo del documento
\usepackage[spanish]{babel} %Permite definir el idioma del dcumento
\usepackage{graphicx} %Permite exportar imagenes en formato eps
\usepackage{url} %Tipo de fuente para correos y paginas
\usepackage{pgf}
\usepackage{fleqn}
\usepackage{amssymb}
\usepackage{amsmath}
\usepackage{fancyvrb}
\usepackage{makeidx}
\usepackage{colortbl} %Permite colocar colores a las tablas
\usepackage{booktabs}
\usepackage[final]{pdfpages}
%%%%%%%%%%
%Margenes%
%%%%%%%%%%
\parskip 1mm %Espacio entre parrafos

\setlength{\topmargin}{0pt}
\topmargin      0.5cm
\oddsidemargin	0.1cm  % Ancho Letter 21,59cm
\evensidemargin 0.5cm  % Alto  Letter 27,81cm
\textwidth	17cm%15.5cm
\textheight	21.0cm
\headsep	4 mm
\parindent	0.5cm
%%%%%%%%%%%%%%%%%%%%%%
%Estilo del documento%
%%%%%%%%%%%%%%%%%%%%%%
\pagestyle{fancyplain}

%%%%%%%%%%%%%%%%%%%%%%%%%%%%%%%%%%%%%%%%%%%
%Fancyheadings. Top y Bottom del documento%
%%%%%%%%%%%%%%%%%%%%%%%%%%%%%%%%%%%%%%%%%%%
% Recuerde que en este documento la portada del documento no posee
% numeracion, pero de igual manera llamaremos a esa primera pagina la numero
% 1, y la que viene la dos. Esto es para tener una idea de las que
% llamaremos pares e impares
\lhead{Investigación de Operaciones I} %Parte superior izquierda
\rhead{\bf \it Tarea 3} %Parte superior derecha
\lfoot{\it } %Parte inferior izquierda. \thepage indica
% el numero de pagina
\cfoot{} %Parte inferior central
\rfoot{\bf \thepage} %Parte inferior derecha
\renewcommand{\footrulewidth}{0.4pt} %Linea de separacion inferior

\newcommand{\primaria}[1]{
	\textbf{\underline{#1}}
}

\newcommand{\foranea}[1]{
	\textbf{\textsl{#1}}
}

\newcommand{\primyfor}[1]{
	\underline{\foranea{#1}}
}

\makeatletter
\newcommand\subsubsubsection{\@startsection {paragraph}{1}{\z@}%
                                   {-3.5ex \@plus -1ex \@minus -.2ex}%
                                   {1.5ex \@plus.2ex}%
                                   {\normalfont\bfseries}}
                       
                                
                                 
\newcommand\subsubsubsubsection{\@startsection {subparagraph}{1}{\z@}%
                                   {-3.5ex \@plus -1ex \@minus -.2ex}%
                                   {1.5ex \@plus.2ex}%
                                  
                                   {\normalfont\bfseries}}


\makeatother
 

\begin{document}
\title{Investigación de Operaciones I \\ \begin{Large}Tarea 3\end{Large}} 
\author{Victor Gonzalez (2.773.029-9)
\and Cesar Muñoz (2.973.053-0)}
\date{\today}
\maketitle


\section{Maximizando el lucro}
\subsection{¿Cuál es la mínima cantidad de proveedores requerida para cumplir con las ganancias pedido y quiénes serían?}
\subsubsection{Solución no óptima}
\subsubsection{Solución optimizada}

\subsection{Solución con todos los vendedores}
%\input{./src/tarea2_parte1_costo.ltx}


\section{NutraFat Alimentos}
\subsection{Condiciones de tiempo}
\subsubsection{¿Cual es el recorrido óptimo que debe hacer el chofer?}
Para resolver este problema, se hizo un matriz con todas los tiempos entre los destinos, pero dada la complejidad por la cantidad de conexiones, se recurrió al uso de \textit{lp\_solve}.

En primera instancia se programó el siguiente código:

\begin{verbatim}
/* Función objetivo: minimizar tiempos*/
min: 999999 x11 + 999999 x22 + 999999 x33 + 999999 x44 + 999999 x55 + 999999 x66
 + 999999 x77 + 999999 x88 + 999999 x99 + 999999 x14 + 999999 x41 + 999999 x18 +
999999 x81 + 999999 x19 + 999999 x91 + 999999 x25 + 999999 x52 + 999999 x26 +
999999 x62 + 999999 x28 + 999999 x82 + 999999 x36 + 999999 x63 + 999999 x37 +
999999 x73 + 999999 x39 + 999999 x93 + 999999 x46 + 999999 x64 + 999999 x48 +
999999 x84 + 999999 x59 + 999999 x95 + 999999 x67 + 999999 x76 + 999999 x68 +
999999 x86 + 10 x12 + 10 x21 + 24 x13 + 24 x31 + 30 x15 + 30 x51 + 15 x16 +
15 x61 + 12 x17 + 12 x71 + 35 x23 + 35 x32 + 17 x24 + 17 x42 + 27 x34 + 27 x43
+ 25 x35 + 25 x53 + 20 x38 + 20 x83 + 12 x45 + 12 x54 + 18 x47 + 18 x74 + 15 x49
+ 15 x94 + 20 x56 + 20 x65 + 24 x57 + 24 x75 + 40 x58 + 40 x85 + 21 x69 + 21 x96
+ 5 x78 + 21 x79 + 15 x87 + 23 x89 + 15 x97 + 7 x98; 

/* Método hungaro */
/* Máximo una asignación por fila */
x11 + x12 + x13 + x14 + x15 + x16 + x17 + x18 + x19 = 1;
x21 + x22 + x23 + x24 + x25 + x26 + x27 + x28 + x29 = 1;
x31 + x32 + x33 + x34 + x35 + x36 + x37 + x38 + x39 = 1;
x41 + x42 + x43 + x44 + x45 + x46 + x47 + x48 + x49 = 1;
x51 + x52 + x53 + x54 + x55 + x56 + x57 + x58 + x59 = 1;
x61 + x62 + x63 + x64 + x65 + x66 + x67 + x68 + x69 = 1;
x71 + x72 + x73 + x74 + x75 + x76 + x77 + x78 + x79 = 1;
x81 + x82 + x83 + x84 + x85 + x86 + x87 + x88 + x89 = 1;
x91 + x92 + x93 + x94 + x95 + x96 + x97 + x98 + x99 = 1;

/* Máximo una asignación por columna */
x11 + x21 + x31 + x41 + x51 + x61 + x71 + x81 + x91 = 1;
x12 + x22 + x32 + x42 + x52 + x62 + x72 + x82 + x92 = 1;
x13 + x23 + x33 + x43 + x53 + x63 + x73 + x83 + x93 = 1;
x14 + x24 + x34 + x44 + x54 + x64 + x74 + x84 + x94 = 1;
x15 + x25 + x35 + x45 + x55 + x65 + x75 + x85 + x95 = 1;
x16 + x26 + x36 + x46 + x56 + x66 + x76 + x86 + x96 = 1;
x17 + x27 + x37 + x47 + x57 + x67 + x77 + x87 + x97 = 1;
x18 + x28 + x38 + x48 + x58 + x68 + x78 + x88 + x98 = 1;
x19 + x29 + x39 + x49 + x59 + x69 + x79 + x89 + x99 = 1;
\end{verbatim}

\{1,2,3,4,5,6,7,8,9\} = \{HQ, AL, DAN, JP, USM, NIC, PUCV, YON, NIL\}

x12 representa el camino entre HQ y AL.

\textit{\textbf{Nota:} no es necesario indicar la naturaleza de las variables dada las restricciones del problema.}
\\

El resultado de esto nos indica que la ruta a seguir por el repartidor es:
$HQ \to NIC \to HQ$ y $DAN \to USM \to JP \to NIL \to AL \to PUCV \to YON \to DAN$. Con un tiempo total del viaje
de 107 minutos.

Podemos identificar inmediatamente que hay un loop entre lo que es HQ y NIC, por lo que agregaremos una nueva restricción, para quebrar ese loop, de manera que no hay camino de $HQ \to NIC$ o no hay camino de $NIC \to HQ$, por lo cual se agrega la restricción: \verb+x16 = 0;+ y se modifica la F.O. a $... + 999999 x_{16} + ...$. Por otro lado se prueba también con la alternativa de: \verb+x61 = 0;+ y se modifica la F.O. a $... + 999999 x_{61} + ...$.

Entre las 2 alternativas, la que más conviene es el segundo caso, ya que su función objetivo es de solo 109 minutos, comparados con los 111 de el caso cuando se rompe $HQ \to NIC$. El camino generado es: 
$JP \to USM \to JP$ y $HQ \to NIC \to NIL \to AL \to PUCV \to YON \to DAN \to HQ$.

Esto nuevamente entrega un loop, por lo que probaremos rompiendo $JP \to USM$ y $USM \to JP$. La opción que más combiene es:

Esto nos entrega el siguiente recorrido del motociclista: $HQ \to NIC \to USM \to JP \to NIL \to AL \to PUCV \to YON \to DAN \to HQ$, con un tiempo de recorrido total de 111 minutos.

\subsubsection{¿Tiene otras posibilidades de rutas?}
Para verificar otras posibles rutas, debemos forzar al cálculo que pase por caminos más económicos. En nuestro caso, el camino que más tarda es entre $DAN \to HQ$, por lo que queremos mejorar esto, para lo cual se añade una nueva restricción que impida que se pase por ahi, simplemente añadiendo ese camino como inexistente en esa dirección.
\subsubsection{¿Grafo de recorrido?}

\subsection{Condiciones de costo}
\subsubsection{¿Cual es la ruta más económica para el chofer?}


\bibliographystyle{alpha}
\bibliography{bibbase}

% referencias

\end{document}
