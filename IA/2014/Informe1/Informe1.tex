\documentclass[letter, 10pt]{article}
\usepackage[utf8]{inputenc}
\usepackage[spanish]{babel}
\usepackage{amsfonts}
\usepackage{amsmath}
\usepackage[dvips]{graphicx}
\usepackage{graphicx} %Permite exportar imagenes en formato eps
\usepackage{url} %Tipo de fuente para correos y paginas
\usepackage{pgf}
\usepackage{fleqn}
\usepackage{amssymb}
\usepackage{amsmath}
\usepackage{fancyvrb}
\usepackage{makeidx}
\usepackage{colortbl} %Permite colocar colores a las tablas
\usepackage{multirow}
\usepackage{booktabs}
\usepackage{moreverb}
\usepackage{rotating}
\usepackage[final]{pdfpages}
\usepackage[top=3cm,bottom=3cm,left=3.5cm,right=3.5cm,footskip=1.5cm,headheight=1.5cm,headsep=.5cm,textheight=3cm]{geometry}


\begin{document}
\title{Inteligencia Artificial \\ \begin{Large}Estado del Arte: Progressive Party Problem\end{Large}}
\author{Victor Andres Roberto Gonzalez Rodriguez}
\date{\today}
\maketitle


%--------------------No borrar esta secci\'on--------------------------------%
\section*{Evaluación}

\begin{tabular}{ll}
Resumen (5\%): & \underline{\hspace{2cm}} \\
Introducción (5\%):  & \underline{\hspace{2cm}} \\
Definición del Problema (10\%):  & \underline{\hspace{2cm}} \\
Estado del Arte (35\%):  & \underline{\hspace{2cm}} \\
Modelo Matemático (20\%): &  \underline{\hspace{2cm}}\\
Conclusiones (20\%): &  \underline{\hspace{2cm}}\\
Bibliografía (5\%): & \underline{\hspace{2cm}}\\
 &  \\
\textbf{Nota Final (100\%)}:   & \underline{\hspace{2cm}}
\end{tabular}
%---------------------------------------------------------------------------%
\vspace{2cm}


\begin{abstract}
El Progressive Party Problem (PPP), es un complejo problema de optimización combinatorial sujeto a restricciones propuesto en el año 1995. El objetivo del problema consiste en encontrar la asignación de recursos de las variables para una cantidad determinada de periodos de tiempo, siempre y cuando se satisfagan un serie de restricciones. Hasta el momento, se han utilizado tres métodos para tratar de resolver el problema: mediante Programación Lineal Entera o Mixta, mediante Programación con Restricciones, y mediante Búsqueda Local (esta última con distintas variaciones). Los resultados al tratar de resolver este problema muestran que la Programación Lineal Entera o Mixta tiene un mal desempeño comparado con la Programación con Restricciones o con Busqueda Local. Finalmente, se presenta un modelo matemático que describe el problema de manera estandar.
\end{abstract}

\section{Introducción}
En el presente documento se analizará en detalle el Problema de la Fiesta Progresiva, o Progressive Party Problem (PPP), su historia, el estado del arte y como han evolucionado los métodos resolutivos para este problema.\\

Antes de continuar, es necesario detallar qué es el Progressive Party Problem. El problema consiste basicamente en lo siguiente: se tiene una fiesta de yates donde cada anfitrión (entiendase el dueño del bote), dispone de su bote de manera que cada asistente a la fiesta pueda pasar por todos los botes, y además, los asistentes van cambiandose de yate cada cierto tiempo (normalmente cada 30 minutos). Todo esto se debe lograr sin superar la capacidad máxima de pasajeros, y sin visitar el mismo bote más de una vez.

En esencia, en esa dinámica se modela el Progressive Party Problem, la cual es un problema combinatorio muy complejo.\\

En este documento se detallará a cabalidad los detalles del problema, tales como sus variables, las restricciones y el objetivo, de manera que se pueda apreciar de manera estandarizada el problema más allá de las modificaciones que proponen distintos autores. Además, veremos en qué posición se encuentra la ciencia de la computación en la actualidad para resolver este problema. Finalmente, se presentará un modelo matemático que representará todas sus variables, restricciones y cualquier función que se estime necesaria.\\

Con este documento el lector podrá quedar completamente interiorizado y actualizado del Progressive Party Problem.

\section{Definición del Problema}
\cite{FirstPublication}

\section{Estado del Arte}


\section{Modelo Matemático}

\section{Conclusiones}
Conclusiones RELEVANTES del estudio realizado.

\section{Bibliograf\'ia}
Indicando toda la informaci\'on necesaria de acuerdo al tipo de documento revisado. Todas las referencias deben ser 
citadas en el documento.
\bibliographystyle{plain}
\bibliography{Referencias}

\end{document} 