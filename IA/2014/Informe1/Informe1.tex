\documentclass[letter, 10pt]{article}
\usepackage[utf8]{inputenc}
\usepackage[spanish]{babel}
\usepackage{amsfonts}
\usepackage{amsmath}
\usepackage[dvips]{graphicx}
\usepackage{graphicx} %Permite exportar imagenes en formato eps
\usepackage{url} %Tipo de fuente para correos y paginas
\usepackage{pgf}
\usepackage{fleqn}
\usepackage{amssymb}
\usepackage{amsmath}
\usepackage{fancyvrb}
\usepackage{makeidx}
\usepackage{colortbl} %Permite colocar colores a las tablas
\usepackage{multirow}
\usepackage{booktabs}
\usepackage{moreverb}
\usepackage{rotating}
\usepackage{hyperref}
\usepackage[final]{pdfpages}
\usepackage[top=3cm,bottom=3cm,left=3.5cm,right=3.5cm,footskip=1.5cm,headheight=1.5cm,headsep=.5cm,textheight=3cm]{geometry}


\begin{document}
\title{Inteligencia Artificial \\ \begin{Large}Estado del Arte: Progressive Party Problem\end{Large}}
\author{Victor Andres Roberto Gonzalez Rodriguez}
\date{\today}
\maketitle


%--------------------No borrar esta secci\'on--------------------------------%
\section*{Evaluación}

\begin{tabular}{ll}
Resumen (5\%): & \underline{\hspace{2cm}} \\
Introducción (5\%):  & \underline{\hspace{2cm}} \\
Definición del Problema (10\%):  & \underline{\hspace{2cm}} \\
Estado del Arte (35\%):  & \underline{\hspace{2cm}} \\
Modelo Matemático (20\%): &  \underline{\hspace{2cm}}\\
Conclusiones (20\%): &  \underline{\hspace{2cm}}\\
Bibliografía (5\%): & \underline{\hspace{2cm}}\\
 &  \\
\textbf{Nota Final (100\%)}:   & \underline{\hspace{2cm}}
\end{tabular}
%---------------------------------------------------------------------------%
\vspace{2cm}


\begin{abstract}
El Progressive Party Problem (PPP), es un complejo problema de optimización combinatorial sujeto a restricciones propuesto en el año 1995. El objetivo del problema consiste en una fiesta de yates, donde los invitados deben ser capaces de recorrer todos los yates anfitriones, cambiandose de yate cada cierto tiempo y cumpliendo ciertas restricciones. Hasta el momento, se han utilizado tres métodos para tratar de resolver el problema: mediante Programación Lineal Entera o Mixta, mediante Programación con Restricciones, y mediante Búsqueda Local (esta última con distintas variaciones). Los resultados al tratar de resolver este problema muestran que la Programación Lineal Entera o Mixta tiene un mal desempeño comparado con la Programación con Restricciones o con Busqueda Local. Finalmente, se presenta un modelo matemático que describe el problema de manera estandar.
\end{abstract}

\section{Introducción}
En el presente documento se analizará en detalle el Problema de la Fiesta Progresiva, o Progressive Party Problem (PPP), su historia, el estado del arte y como han evolucionado los métodos resolutivos para este problema.

Antes de continuar, es necesario detallar qué es el Progressive Party Problem. El problema consiste basicamente en lo siguiente: se tiene una fiesta de yates donde cada anfitrión (entiendase el dueño del bote), dispone de su bote de manera que cada asistente a la fiesta pueda pasar por todos los botes, y además, los asistentes van cambiandose de yate cada cierto tiempo (normalmente cada 30 minutos). Todo esto se debe lograr sin superar la capacidad máxima de pasajeros, y sin visitar el mismo bote más de una vez.

En esencia, en esa dinámica se modela el Progressive Party Problem, la cual es un problema combinatorio muy complejo.\\

En este documento se detallará a cabalidad los detalles del problema, tales como sus variables, las restricciones y el objetivo, de manera que se pueda apreciar de manera estandarizada el problema más allá de las modificaciones que proponen distintos autores. Además, veremos en qué posición se encuentra la ciencia de la computación en la actualidad para resolver este problema. Finalmente, se presentará un modelo matemático que representará todas sus variables, restricciones y cualquier función que se estime necesaria.

Con este documento el lector podrá quedar completamente interiorizado y actualizado del Progressive Party Problem.

\section{Definición del Problema}

El Progressive Party Problem se formula de la siguiente manera: considérese una fiesta durante una reunión de yates. Existen $n$ botes y sus tripulaciones. Un número de determinado de botes es escogido para ser bote anfitrión: estos serán los botes donde se realizará la fiesta, donde otras tripulaciones visitarán al bote en intervalos de tiempo $t=1..T$ de media hora cada uno. El número total de periodos $T$ viene dado. Las \textit{tripulaciones visitantes} se mueven de un bote anfitrión a otro, y no pueden visitar un bote más de una vez. Los botes tienen una cierta capacidad limitada de tripulación a bordo, la cual no debe ser superada: esta es la capacidad de visitas de un bote anfitrión. Además, las \textit{tripulaciones visitantes} no pueden encontrarse más de una vez con la misma \textit{tripulación visitante}.

Lo que se busca es una calendarización que minimice la cantidad de botes anfitriones. A partir de lo recopilado en \cite{FirstPublication,PPPAsMIP,WalserThesis,LocalSearch}, el problema se define\footnote{En ningun lado existe un modelo estándar de PPP, pero en todas las publicaciones se respeta la formulación hecha por la publicación original \cite{FirstPublication}, la cual es respetada en la definición utilizada en este documento.} como: \\

\subsection{Parámetros}
\begin{itemize}
\item $T$ (numero de periodos), el cual define cuantos intervalos de tiempos se manejarán en el problema.
\item $n$ (cantidad de botes), el cual dice cuantos botes existen disponibles para realizar la fiesta.\footnote{Recordar que se busca minimizar la cantidad de botes que se utilizarán finalmente, por lo que es posible que no se utilizen todos los botes disponibles como anfitriones.}
\item $w_i$ (tamaño de la tripulación del bote), el cual indica cuantos tripulantes quedan fijos en el bote $i$.
\item $p_i$ (capacidad de tripulación del bote), el cual define la capacidad máxima de personas que puede soportar un bote $i$ sin hundirse.
\end{itemize}

\subsection{Objetivos}
\begin{itemize}
\item Todas las tripulaciones visitantes deben visitar todos los yates.
\item Promover que los visitantes tengan la mayor cantidad de interacción social.
\item \textbf{Minimizar la cantidad de botes anfitriones.}
\item Satisfacer las necesidades y restricciones de la fiesta.
\end{itemize}

Todos estos objetivos serán pulidos y ajustados a un listado de requerimientos
que buscan que la fiesta se produzca sin accidentes (respetar limites de capacidad), que se mantenga socialmente activa (evitando que las tripulaciones visitantes se encuentren más de una vez), entre otros requerimientos.

\subsection{Restricciones}
Las restricciones generales de este problema son bastante simples y directas. Pero más adelante se podrá ver que existen varias representaciones para estas restricciones, aunque todas buscan cubrir los mismos requerimientos.

\begin{itemize}
\item Las fiestas solo se pueden realizar en los botes anfitriones.
\item La capacidad de un bote no puede ser excedida.
\item Las tripulaciones no pueden estar sin hacer nada: o estan visitando botes anfitriones, o bien ellos mismos son anfitriones.
\item Las tripulaciones no pueden visitar el mismo bote mas de una vez.
\item Las tripulaciones visitantes no pueden cruzarse mas de una vez.
\end{itemize}

\subsection{Problemas Similares}
El Progressive Party Problem es considerado un problema particular de Timetabling (programación de horarios), por lo que problemas que buscan calendarizar eventos y optimizarlos bajo algun criterio, son considerados problemas similares.

\subsection{Variantes}
En el articulo \cite{FirstPublication} y \cite{PPPAsMIP} se muestran algunas variantes que buscan reducir el numero de ecuaciones (y por ende el espacio de búsqueda), del problema. Algunas de esos cambios son:

\begin{itemize}
\item Se introduce una variable binaria adicional que indica si una tripulación $j$ y $j'$ visitan a un bote $i$ en el mismo intervalo de tiempo determinado.
\item Se introduce una variable entera que indica si un bote fue visitado por una tripulación en un instante determinado (variable con penalizaciones).
\end{itemize}

\section{Estado del Arte}
El \textit{``Progressive Party Problem''} (PPP) fue postulado por primera vez y resuelto heurísticamente por Peter Hubbard, el cual es miembro del \textit{``Sea Wych Owners Association''} y del Departamento de Matemáticas de la Universidad de Southhapton. Peter Hubbard era el organizador de una reunión de yates, donde debía, entre muchas otras cosas, organizar una fiesta para los participantes de los yates sobre los yates. \cite{ProblemDescription}

P. Hubbard logró dar con la solución al problema heurísticamente (buscando manualmente), pero pensó que sería interesante ver si es que se podría encontrar una mejor solución y más optima. Le sugirió este problema a H. Paul Williams, el cual formuló el problema bajo \textit{Programación Entera Mixta}, mientras que su colega Sally Brailsford trataba de resolver el problema con software comercial de Programación Entera Mixta. Por otro lado, Barbara Smith usó técnicas de \textit{Programación con Restricciones} (es decir, se formuló como CSP). Estos dos enfoques dieron paso al primer articulo publicado sobre el Progressive Party Problem. \cite{FirstPublication}\\

La publicación de P. Hubbard y sus colegas en 1995 \cite{FirstPublication} (la cual fue incluida posteriormente en otra publicacion en 1996), se enfocó principalmente en detallar las diferencias empíricas que se obtuvieron al desarrollar la solución del PPP en Programación Entera (PE) y en Programación con Restricciones (PR). El resultado fue desastrozo para la PE, debido a que este tipo de problemas de Timetabling eran problemas comunmente resueltos bajo ese paradigma de programación, pero que sin embargo fue bastante alentador para la PR. Esta situación fue lo que inició la interesante ``guerra'' entre la Programación Entera y la Programación con Restricciones en torno al PPP.

En 1997-1998, un nuevo autor \cite{PseudoLocal,WalserThesis}, postula que las nuevas técnicas implementadas en la Programación Entera Mixta (MIP) son mucho mejores, por lo que utiliza MIP's para  resolver el problema del PPP, lo cual antes había sido desastrozo.

Ya recién en 1999 se utilizó un nuevo acercamiento al PPP, esta vez utilizando Local Search (Tabu Search, Simulated Annealing), las cuales eran eficientes, pero no siempre óptimas\footnote{Este es un problema común al utilizar estos métodos}.

De ahí en adelante, la literatura sobre el tema decayó, pero distintas implementaciones en distintos programas, metodos y representaciónes han tenido publicaciones. \cite{CustomizingSearch,ProgressOnThePartyProblem}

\subsection{Métodos Utilizados Para Resolverlo}
Para resolver el PPP, se menciona repetidamente en varios de los articulos que se utiliza Programación Entera o Mixta, para los cuales se utiliza el software GAMS\footnote{\url{http://www.gams.com/}}, o tambien se menciona el uso de Programación con Restricciones\cite{FirstPublication}. Pero en los articulos más recientes, se hace uso explícito de métodos de búsqueda local, como Tabu Search y Simulated Annealing.

\subsubsection{Programación Entera o Mixta}
Es un modelo de programación que contiene restricciones y una función objetivo. En el caso de la Programación Entera, todas las variables deben ser enteras, en cambio para la Programación Mixta, las variables pueden tomar valores enteros como tambien binarios.

Este modelo de programación resulta bastante útil en especial para los problemas de tipo Timetabling, ya que las restricciones y la función objetivo de este tipo de problemas tienden a ser altamente representables mediante formulas matemáticas sencillas.

Software como GAMS, son altamente utilizados en la resolución de este tipo de problemas.

\subsubsection{Programación con Restricciones}
La programación con restricciones, es otro modelo de programación donde las variables se representan mediante las restricciones que las relacionan. En este tipo de paradigma, las restricciones de los modelos difieren de los modelos tradicionales de programación, ya que en vez de ejecutarse pasos o procedimientos, la Programación con Restricciones especifica las propiedades de las posibles soluciones. Este enfoque hace que la Programación con Restricciones sea parte del mundo de la Programación Declarativa.

Un software bastante conocido en el mundo académico que utiliza este modelo de programación es Prolog, pero en los estudios del PPP se menciona CHIP.

\subsubsection{Simulated Annealing}
Simulated Annealing es un procedimiento de búsqueda local para explorar el espacio de soluciones más allá del óptimo local, el cual busca reducir el problema de los algoritmos de búsqueda local, aceptando soluciones de ``peor calidad''.

Al igual que otros algoritmos de búsqueda local, Simulated Annealing necesita una solución inicial, y comienza explorando por el vecindario de todos los optimos locales. A medida que el tiempo pasa, una función de temperatura va disminuyendo, lo cual obliga al algoritmo a explotar más a los óptimos locales en vez de explorar.

Este método necesita definir una representación de los datos, un movimiento para explotar, una función de evaluación que indica que tan buena es una solución, una temperatura inicial y su respectiva función de decaimiento, y el número de iteraciones máximas para el algoritmo.

\subsubsection{Tabu Search}
Tabu Search es un algoritmo de búsqueda local, que busca explotar y explorar el espacio de búsqueda, de acuerdo a una estructura de memoria adaptativa y flexible.

Al igual que Simulated Annealing, Tabu Search necesita una solución inicial para comenzar a explorar, y va almacenando en memoria la lista tabú, para evitar que se produzcan ciclos, y además, puede aceptar soluciones peores a la actual, pero siempre mantiene en memoria a la mejor solución.

Este método necesita definir una representación de los datos, un movimiento para explotar, una función de evaluación que indica que tan buena es una solución, el tamaño de la lista tabú, y el número de iteraciones máximas para el algoritmo.

\subsection{Heurísticas y Metaheurísticas}
\cite{LocalSearch} heuristica Min-Conflict

\cite{PPPAsMIP} heuristica Time Staged

\cite{FirstPublication} Ordenamiento
\subsection{Resultados}


\cite{LocalSearch}
\begin{tabular}{|l||c|c|c|c|}
\hline
problem & ILP & CP1 & CP2 & LS \\ \hline
$P_6$ & fail & 27 min. & a few sec. & $< 1$ s. \\ \hline
$P_7$ & fail & 28 min. & a few sec. & $< 1$ s. \\ \hline
$P_8$ & fail & fail & a few sec. & $1$ s. \\ \hline
$P_9$ & fail & fail & hours & $4$ s. \\ \hline
$P_10$ & fail & fail & fail & fail \\ \hline
\end{tabular}

\cite{PPPAsMIP}
\begin{tabular}{|p{0.8cm}|l|l|p{2cm}|p{1.2cm}|p{0.7cm}|p{0.7cm}|}
\hline
time stage & nº equations & nº variables & nº non-zero elem. & discrete vars. & gen. time & sol. time \\ \hline
1 & $38830$ & $2620$ & $114130$ & $1758$ & $0.73$ & $1.62$ \\ \hline
... & ... & ... & ... & ... & ... & ... \\ \hline
(7) & $245470$ & $77500$ & $322876$ & $1722$ & $3.42$ & $5.50$ \\ \hline
\end{tabular}


original sin resultados para el problema completo

+Metodos utilizados hasta el momento
+Mejores algoritmos utilizados
+Que representaciones han tenido mejores resultados
+Tendencia actual
+Tipos ed movimientos
+heuristicas
+Metodos completos
+Graficos comparativos o explicativos

\section{Modelo Matemático}
Existen varias formas de plantar este modelo, pero se busca preservar la simplicidad y semántica que respeten la originalidad del problema.

Es por eso que las representaciones y restricciones aquí definidas, están basadas en la publicación original del Progressive Party Problem \cite{FirstPublication}, complementada con la información disponible en \cite{CustomizingSearch}, y respaldada por la publicación \cite{PPPAsMIP}.

\section{Conclusiones}
Conclusiones RELEVANTES del estudio realizado.

\section{Bibliografía}
\bibliographystyle{plain}
\bibliography{Referencias}

\end{document} 